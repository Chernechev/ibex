\documentclass[]{article}
\usepackage{lmodern}
\usepackage{amssymb,amsmath}
\usepackage{ifxetex,ifluatex}
\usepackage{fixltx2e} % provides \textsubscript
\ifnum 0\ifxetex 1\fi\ifluatex 1\fi=0 % if pdftex
  \usepackage[T1]{fontenc}
  \usepackage[utf8]{inputenc}
\else % if luatex or xelatex
  \ifxetex
    \usepackage{mathspec}
    \usepackage{xltxtra,xunicode}
  \else
    \usepackage{fontspec}
  \fi
  \defaultfontfeatures{Mapping=tex-text,Scale=MatchLowercase}
  \newcommand{\euro}{€}
\fi
% use upquote if available, for straight quotes in verbatim environments
\IfFileExists{upquote.sty}{\usepackage{upquote}}{}
% use microtype if available
\IfFileExists{microtype.sty}{%
\usepackage{microtype}
\UseMicrotypeSet[protrusion]{basicmath} % disable protrusion for tt fonts
}{}
\usepackage{color}
\usepackage{fancyvrb}
\newcommand{\VerbBar}{|}
\newcommand{\VERB}{\Verb[commandchars=\\\{\}]}
\DefineVerbatimEnvironment{Highlighting}{Verbatim}{commandchars=\\\{\}}
% Add ',fontsize=\small' for more characters per line
\newenvironment{Shaded}{}{}
\newcommand{\KeywordTok}[1]{\textcolor[rgb]{0.00,0.44,0.13}{\textbf{{#1}}}}
\newcommand{\DataTypeTok}[1]{\textcolor[rgb]{0.56,0.13,0.00}{{#1}}}
\newcommand{\DecValTok}[1]{\textcolor[rgb]{0.25,0.63,0.44}{{#1}}}
\newcommand{\BaseNTok}[1]{\textcolor[rgb]{0.25,0.63,0.44}{{#1}}}
\newcommand{\FloatTok}[1]{\textcolor[rgb]{0.25,0.63,0.44}{{#1}}}
\newcommand{\CharTok}[1]{\textcolor[rgb]{0.25,0.44,0.63}{{#1}}}
\newcommand{\StringTok}[1]{\textcolor[rgb]{0.25,0.44,0.63}{{#1}}}
\newcommand{\CommentTok}[1]{\textcolor[rgb]{0.38,0.63,0.69}{\textit{{#1}}}}
\newcommand{\OtherTok}[1]{\textcolor[rgb]{0.00,0.44,0.13}{{#1}}}
\newcommand{\AlertTok}[1]{\textcolor[rgb]{1.00,0.00,0.00}{\textbf{{#1}}}}
\newcommand{\FunctionTok}[1]{\textcolor[rgb]{0.02,0.16,0.49}{{#1}}}
\newcommand{\RegionMarkerTok}[1]{{#1}}
\newcommand{\ErrorTok}[1]{\textcolor[rgb]{1.00,0.00,0.00}{\textbf{{#1}}}}
\newcommand{\NormalTok}[1]{{#1}}
\usepackage{longtable,booktabs}
\ifxetex
  \usepackage[setpagesize=false, % page size defined by xetex
              unicode=false, % unicode breaks when used with xetex
              xetex]{hyperref}
\else
  \usepackage[unicode=true]{hyperref}
\fi
\hypersetup{breaklinks=true,
            bookmarks=true,
            pdfauthor={},
            pdftitle={},
            colorlinks=true,
            citecolor=blue,
            urlcolor=blue,
            linkcolor=magenta,
            pdfborder={0 0 0}}
\urlstyle{same}  % don't use monospace font for urls
\setlength{\parindent}{0pt}
\setlength{\parskip}{6pt plus 2pt minus 1pt}
\setlength{\emergencystretch}{3em}  % prevent overfull lines
\setcounter{secnumdepth}{0}

\date{}
\usepackage[left=1.0in,right=1.0in,top=1.0in,bottom=1.0in]{geometry}
\usepackage[]{mathptmx}
\usepackage{sectsty}
\usepackage{verbatim}
\usepackage{ulem}
\usepackage{breakurl}
\usepackage{ragged2e}
\usepackage{hyperref}
\hypersetup{colorlinks=true,urlcolor=blue,linkcolor=blue}
\sectionfont{\Large \bf}
\subsectionfont{\large \bf}
\subsubsectionfont{\normalsize \bf}

\begin{document}


\date{}
\author{Alex Drummond}
\title{Ibex 0.3.6 Manual}

\maketitle

\tableofcontents

\textbf{Check out the printable \href{manual.pdf}{PDF version} of this
manual.}

\section{Versions}\label{versions}

This documentation covers 0.3.7. The 0.1.x versions of Webspr come with
a README containing full documentation. 0.2.x versions are documented in
another page on the wiki. The latest version of this document can be
found at
\url{https://github.com/addrummond/ibex/blob/master/docs/manual.md}

See the end of this document for the changelog.

Prior to version 0.3 this software was called ``webspr'', not ``Ibex''.

\section{Requirements}\label{requirements}

\begin{itemize}
\itemsep1pt\parskip0pt\parsep0pt
\item
  Python \textgreater{}= 2.3. (Python 3000 not currently supported.)
\end{itemize}

\section{Introduction}\label{introduction}

Ibex (``Internet Based EXperiments'') allows you to run various kinds of
psycholinguistic experiment online. The system is modular and the range
of possible experiments is steadily growing. Ibex uses JavaScript and
HTML only, and doesn't require the use of plugins such as Java or Flash.

\section{Setting up the server}\label{setting-up-the-server}

\textbf{Note:} If you are using the Ibex Farm, you can skip to the
``Basic Concepts'' section below. For most applications you will only
need to modify the file \texttt{example\_data.js}. (The Ibex Farm allows
you to modify/upload some other files, but this rarely necessary.)

There are two ways of running the server: either using the stand-alone
toy HTTP server, or as a CGI process. By default, the server runs in
stand-alone mode. To start it, change to the \texttt{www} directory and
execute server.py:

\begin{verbatim}
    python server.py
\end{verbatim}

This will start the server on port 3000 by default (so the experiment
will be at \url{http://localhost:3000/experiment.html}).

The stand-alone server is limited to serving files relating to Ibex, so
it is unlikely to be useful for real work (unless perhaps you hide it
behind a proxy). However, it is useful for testing experimental designs
without setting up a real HTTP server, and for running experiments
offline.

The server can be configured by editing \texttt{server\_conf.py}. Each
of the options is commented. Change the \texttt{SERVER\_MODE} variable
to determine whether the server runs in stand-alone mode or as a CGI
application. When \texttt{SERVER\_MODE} is set to ``cgi'',
\texttt{server.py} will work as a standard CGI program; when it is set
to ``toy'', the server will run in stand-alone mode. (The value
``paste'' has the same effect as ``toy'', for backwards compatibility
with 0.2.x version.) The value of the \texttt{PORT} variable determines
the port the server will listen on when operating in stand-alone mode.
As an alternative to editing the configuration file, the \texttt{-m} and
\texttt{-p} command-line options can also be used to set the server mode
and port respectively. Command-line options override those set in
\texttt{server\_conf.py}.

\textbf{New:} Version 0.3 adds some new ways of configuring the server.
These are unlikely to be of interest to most users, but are documented
in the ``New Configuration Methods'' section below.

If running the server as a CGI application, make sure that the web
server can serve up \texttt{experiment.html}, and the other files in the
\texttt{www} dir, as static files. All dynamic requests go through
\texttt{server.py}, so as long as the files in \texttt{www} can be
accessed, and the server recognizes \texttt{server.py} as a CGI
application, everything should work. The ``Step-by-step CGI'' section
below goes through the process of setting up webspr as a CGI app.

The directories \texttt{js\_includes}, \texttt{data\_includes} and
\texttt{css\_includes} contain JavaScript and CSS files required for
particular kinds of experiment. When an HTTP request is made for
\texttt{server.py?include=js}, \texttt{server.py?include=data} or
\texttt{server.py?include=css}, the server concatenates all the files in
the specified directory and serves up the result. Your data file(s) will
live in \texttt{data\_includes}. \textbf{Note:} The server mangles CSS
files by adding a prefix to each class/id name based on the name of the
CSS file; see the section ``CSS mangling'' below for more details.

The server stores some state in the \texttt{server\_state} directory.
This directory is automatically created by the server if it does not
already exist, and in order to start the server from a fresh state you
may simply delete this directory and/or its contents. Currently, the
directory just contains the counter used to alternate latin square
designs (see ``Latin Square Designs'' section below). This counter can
be reset on server startup using the \texttt{-r} option.

The server logs messages in the file \texttt{server.log}. The default is
to store all files in the same directory as \texttt{server.py}
(i.e.~logs, results, the \texttt{server\_state} directory, etc.), but
the working directory of the server can be modified by setting the
variable \texttt{IBEX\_WORKING\_DIR} in \texttt{server\_conf.py}. You
can also set the environment variable of the same name. The value of
this environment variable overrides the value of the variable in
\texttt{server\_conf.py}.

On Linux/Unix/OS X, the server uses file locking to ensure that the
results file and server state remain consistent. Currently, it does not
do so on Windows, so there is a theoretical possibility of the server
state or the results being corrupted if the server is deployed in a
Windows environment (but unless there is very high traffic, it's purely
theoretical).

The stand-alone server has been tested on Windows and OS X (but will
almost certainly work on any system with Python 2.3-2.6). The CGI server
has been tested on OS X and Linux using the lighttpd web server in both
cases.

\subsection{New Configuration Methods}\label{new-configuration-methods}

Version 0.3 of this software introduces two new methods of setting
configuration variables. The first is simply to include the variable
definitions in \texttt{server\_conf.py} directly in \texttt{server.py},
obviating the need to set the \texttt{SERVER\_CONF\_PY\_FILE} variables
in \texttt{server.py}. The second is to specify that \texttt{server.py}
should issue an HTTP GET request to a given URL, with the result
obtained being interpreted as a JSON dictionary specifying the values of
the configuration variables. This mode is configured by setting the
configuration variables \texttt{EXTERNAL\_CONFIG\_URL},
\texttt{EXTERNAL\_CONFIG\_METHOD} and
\texttt{EXTERNAL\_CONFIG\_PASS\_PARAMS}. The first of these variables is
self-explanatory; the second must be set to \texttt{"GET"}; the third
should be set to a boolean, specifying whether or not \texttt{server.py}
should pass a \texttt{dir} paramater in the query string giving the
directory in which \texttt{server.py} resides. Note that the values of
these three variables may be set either in \texttt{server\_conf.py}, or
directly in \texttt{server.py}.

The new configuration methods are used to handle the configuration of
experiments in the Ibex Farm (\url{http://spellout.net/ibexfarm}). They
are unlikely to be of interest to other users.

\subsection{Step-by-step CGI}\label{step-by-step-cgi}

This subsection describes how to set up webspr as a CGI application from
scratch. Since the particulars of different web servers differ, some
parts are necessarily rather vague. There are many possible ways of
organizing the files/directories; the setup described below is just the
simplest.

\begin{enumerate}
\item
  Ensure that the \texttt{www} directory is somewhere within the root
  directory of your HTTP server. You may want to place the entire
  \texttt{webspr} directory within the root directory, or you may want
  to copy the files in \texttt{www} to a new location within the root
  directory. All of these files are ordinary static files except for
  \texttt{server.py}, which needs to be run as a CGI app.
\item
  Decide on a location for the main webspr directory. Depending on the
  details of your setup, you may be able to leave it in place, or you
  may need to copy it to a directory which the HTTP server has
  permission to access (so that \texttt{server.py} may access files in
  this directory when it is executed).
\item
  Edit \texttt{server.py} and change the value of
  \texttt{SERVER\_CONF\_PY\_FILE} to point to the new location of
  server\_conf.py.
\item
  Edit \texttt{server\_conf.py} and change the value of
  \texttt{IBEX\_WORKING\_DIR} to the new location of the main
  \texttt{webspr} dir.
\item
  Edit \texttt{server\_conf.py} and change change the value of
  \texttt{SERVER\_MODE} to \texttt{"cgi"}.
\item
  If the version of Python you wish to use is not the default used by
  the HTTP server, add a \texttt{\#!} line at the beginning of
  \texttt{server.py}.
\item
  You may need to rename \texttt{server.py} (e.g.~some servers may only
  execute CGI scripts with the extension \texttt{.cgi}). If this is the
  case, follow the instructions in the subsection below.
\end{enumerate}

Once these steps are complete, it's just a matter of configuring your
HTTP server correctly. If you are using shared hosting, there is a
reasonably good chance that the HTTP server will already have been
configured to run Python CGI apps, in which case you won't need to do
any additional configuration. There is an example configuration file for
lighttpd included in the distribution (\texttt{example\_lighttpd.conf}).

\subsubsection{Renaming server.py}\label{renaming-server.py}

Ibex will stop working if you rename \texttt{server.py}, since the files
\texttt{experiment.html} and \texttt{overview.html} assume that the
script has this name. (In older versions of ibex there were a few other
files which made this assumption, but as of 0.3-beta14, this has been
changed.)

You can edit \texttt{experiment.html} and \texttt{overview.html}
manually to replace all instances of \texttt{server.py} with the new
name of the script. Alternatively, you can execute the renamed
\texttt{server.py} script with the following options to automatically
generate new \texttt{experiment.html} and \texttt{overview.html} files:

\begin{verbatim}
    python /path/to/renamed/server.py --genhtml /foo/webspr-XXXX/www
\end{verbatim}

This command will produce no output if it executes successfully. Note
that you must run this command \textbf{after} renaming
\texttt{server.py}.

\section{Basic concepts}\label{basic-concepts}

An Ibex experiment is essentially a sequence of \emph{items}. Each item
is itself a sequence of \emph{entities}. To give a concrete example,
each item in a self-paced reading experiment might consist of two
entities: an entity providing a sentence for the subject to step through
word by word, followed by an entity posing a comprehension question.
Some other items and entities would also be present under most
circumstances. For example, there would probably be a ``separator''
entity between each sentence/question pair in order to provide a pause
between sentences. Schematically, the sequence of items would be as
follows (the Wiki insists on displaying this in pretty colors):

\begin{verbatim}
    ITEM 1:
        ENTITY 1: Sentence
        ENTITY 2: Comprehension question
    ITEM 2:
        ENTITY 1: Pause for two seconds
    ITEM 3:
        ENTITY 1: Sentence
        ENTITY 2: Comprehension question
    ITEM 4:
        ENTITY 1: Pause for two seconds
    ITEM 5:
        ENTITY 1: Sentence
        ENTITY 2: Comprehension question
\end{verbatim}

It is not necessary to construct the full sequence of items and entities
manually. Ibex provides tools for ordering items in various ways
(e.g.~for inserting a pause between each item) and for implementing
latin square designs. More on these shortly.

Each entity is an instance of a \emph{controller}, which determines the
kind of experimental task that the entity will present. For example,
there is a \texttt{DashedSentence} controller useful for self-paced
reading and speeded acceptability judgment tasks. Ibex has a modular
design, where each controller is a JavaScript object that follows a
standard interface. This makes it quite easy to add new controllers if
you are familiar with JavaScript/DHTML programming.

Ibex stores results in CSV format. This makes it easy to import results
into spreadsheets, Matlab, R, etc. Each entity may contribute zero, one
\textbf{or more} lines to the results file. The first seven columns of
each line give generic information about the result (e.g.~an MD5 hash
identifying the subject) and the rest give information specific to the
particular element (e.g.~word reading times, comprehension question
answers). The first seven columns are as follows:

\begin{longtable}[c]{p{1in}p{5in}}
\toprule
\textbf{Column} & \textbf{Information}\tabularnewline
\midrule
\endhead
1 & Time results were received (seconds since Jan 1 1970)\tabularnewline
2 & MD5 hash identifying subject. This is based on the subject's IP
address and various properties of their browser. Together with the value
of the first column, this value should uniquely identify each
subject.\tabularnewline
3 & Name of the controller for the entity (e.g.
``DashedSentence'')\tabularnewline
4 & Item number\tabularnewline
5 & Element number\tabularnewline
6 & Type\tabularnewline
7 & Group\tabularnewline
\bottomrule
\end{longtable}

Note that some information pertaining to an entire experiment (e.g.~the
time that the server received the results) is duplicated on each line of
the results. This often makes it easier to parse the results, though it
makes the format much more verbose. The significance of the ``type'' and
``group'' columns will be explained later; they are involved in
specifying the ordering of items.

There are two ways to find out what the values in each column mean. The
``manual'' method is to interpret the first seven columns of each line
as specified above, and then to consult the documentation for the
relevant controllers to determine the format of the subsequent columns.
All of the controllers that are bundled with the Ibex distribution are
fully documented here. An easier method is to look at the comments in
the results file: for each sequence of lines in the same format, the
server adds comments describing the values contained in each column.

\textbf{Important}: A single element my contribute multiple lines to the
results file. For example, if the \texttt{DashedSentence} element is in
``self-paced reading'' mode, it will add a line to the results file for
every word reading time it records. Thus, there is not a one-to-one
correspondence between items/elements and lines.

\textbf{New:} The server now uses a more sophisticated algorithm for
commenting lines in the results file. It is now able to handle cases
where there are repeating patterns of lines in a results file.

\textbf{Newer:} As of 0.3-beta15, there is another commenting algorithm
available, which can be turned on by setting the
\texttt{SIMPLE\_RESULTS\_FILE\_COMMENTS} variable to \texttt{True} in
\texttt{server\_conf.py} (it defaults to \texttt{False}). This algorithm
simply adds a comment for every line in the results file. This simpler
method generally works better than the old algorithm, which tried to be
clever but didn't really succeed. If you are using the Ibex Farm, there
is currently no way of changing this configuration option using the web
interface, but this should be fixed soon.

\section{Format of a data file}\label{format-of-a-data-file}

Data files for Ibex are JavaScript source files, but it is not really
necessary to know JavaScript in order to write one. The most important
part of a data file is the declaration of the ``items'' array, as in the
following example:

\begin{Shaded}
\begin{Highlighting}[]
    \KeywordTok{var} \NormalTok{items = [}

    \NormalTok{[}\StringTok{"filler"}\NormalTok{, }\StringTok{"DashedSentence"}\NormalTok{, \{}\DataTypeTok{s}\NormalTok{: }\StringTok{"Here's a silly filler sentence"}\NormalTok{\}],}
    \NormalTok{[}\StringTok{"filler"}\NormalTok{, }\StringTok{"DashedSentence"}\NormalTok{, \{}\DataTypeTok{s}\NormalTok{: }\StringTok{"And another silly filler sentence"}\NormalTok{\}],}
    \NormalTok{[}\StringTok{"relclause"}\NormalTok{, }\StringTok{"DashedSentence"}\NormalTok{, \{}\DataTypeTok{s}\NormalTok{: }\StringTok{"A sentence that has a relative clause"}\NormalTok{\}],}
    \NormalTok{[}\StringTok{"relclause"}\NormalTok{, }\StringTok{"DashedSentence"}\NormalTok{, \{}\DataTypeTok{s}\NormalTok{: }\StringTok{"Another sentence that has a relative clause"}\NormalTok{\}]}

    \NormalTok{]; }\CommentTok{// NOTE SEMICOLON HERE}
\end{Highlighting}
\end{Shaded}

The ``items'' array is an array of arrays, where each subarray specifies
a single item. In the example above, every item contains a single
element (multiple element items will be covered shortly).

\begin{itemize}
\item
  The first member of each subarray specifies the \emph{type} of the
  item. Types can either be numbers or strings. Although the types in
  the example above have descriptive names, Ibex does not interpret
  these names in any way.
\item
  The second member specifies the controller.
\item
  The third member is an associative array of key/value pairs. This
  array is passed to the controller and is used to customize its
  behavior. In this case, we pass only one option (``s''), which tells
  the \texttt{DashedSentence} controller which sentence it should
  display.
\end{itemize}

Once the items array has been created, Ibex must be told the order in
which the items should be displayed. There are some moderately
sophisticated facilities for creating random orderings and latin square
designs, but for the moment, let's just display the items in the order
we gave them in the array. This can be achieved by adding the following
definition (which won't make much sense yet):

\begin{Shaded}
\begin{Highlighting}[]
    \KeywordTok{var} \NormalTok{shuffleSequence = }\FunctionTok{seq}\NormalTok{(anyType);}
\end{Highlighting}
\end{Shaded}

Suppose we wanted to pair each sentence with a comprehension question.
The easiest way to do this is to add a second element to each item:

\begin{Shaded}
\begin{Highlighting}[]
    \KeywordTok{var} \NormalTok{items = [}

    \NormalTok{[}\StringTok{"filler"}\NormalTok{, }\StringTok{"DashedSentence"}\NormalTok{, \{}\DataTypeTok{s}\NormalTok{: }\StringTok{"Here's a silly filler sentence"}\NormalTok{\},}
               \StringTok{"Question"}\NormalTok{, \{}\DataTypeTok{q}\NormalTok{: }\StringTok{"Is this a filler sentence?"}\NormalTok{, }\DataTypeTok{as}\NormalTok{: [}\StringTok{"Yes"}\NormalTok{, }\StringTok{"No"}\NormalTok{]\}],}
    \NormalTok{[}\StringTok{"filler"}\NormalTok{, }\StringTok{"DashedSentence"}\NormalTok{, \{}\DataTypeTok{s}\NormalTok{: }\StringTok{"And another silly filler sentence"}\NormalTok{\},}
               \StringTok{"Question"}\NormalTok{, \{}\DataTypeTok{q}\NormalTok{: }\StringTok{"Does this sentence have a relative clause?"}\NormalTok{, }\DataTypeTok{as}\NormalTok{: [}\StringTok{"Yes"}\NormalTok{,}\StringTok{"No"}\NormalTok{]],}
    \NormalTok{[}\StringTok{"relclause"}\NormalTok{, }\StringTok{"DashedSentence"}\NormalTok{, \{}\DataTypeTok{s}\NormalTok{: }\StringTok{"A sentence that has a relative clause"}\NormalTok{\},}
                  \StringTok{"Question"}\NormalTok{, \{}\DataTypeTok{q}\NormalTok{: }\StringTok{"Was there movement through [Spec,CP]?"}\NormalTok{, }\DataTypeTok{as}\NormalTok{: [}\StringTok{"Yes"}\NormalTok{, }\StringTok{"No"}\NormalTok{]\}],}
    \NormalTok{[}\StringTok{"relclause"}\NormalTok{, }\StringTok{"DashedSentence"}\NormalTok{, \{}\DataTypeTok{s}\NormalTok{: }\StringTok{"Another sentence that has a relative clause"}\NormalTok{\},}
                  \StringTok{"Question"}\NormalTok{, \{}\DataTypeTok{q}\NormalTok{: }\StringTok{"Was the first word 'frog'?"}\NormalTok{, [}\StringTok{"Yes"}\NormalTok{, }\StringTok{"No"}\NormalTok{]\}]}

    \NormalTok{]; }\CommentTok{// NOTE SEMICOLON HERE}
\end{Highlighting}
\end{Shaded}

As shown above, this is done simply by adding to each item further pairs
of controllers and associative arrays of options. It is rather tiresome
to have to set the \texttt{as} option to {[}``Yes'', ``No''{]} every
time, but this can easily be avoided by specifying this as the default
for the Question controller. To give an example of how defaults for
multiple controllers are specified, let's also set the \texttt{mode}
option of \texttt{DashedSentence} to ``speeded acceptability'':

\begin{Shaded}
\begin{Highlighting}[]
    \KeywordTok{var} \NormalTok{defaults = [}
        \StringTok{"Question"}\NormalTok{, \{}\DataTypeTok{as}\NormalTok{: [}\StringTok{"Yes"}\NormalTok{, }\StringTok{"No"}\NormalTok{]\},}
        \StringTok{"DashedSentence"}\NormalTok{, \{}\DataTypeTok{mode}\NormalTok{: }\StringTok{"speeded acceptability"}\NormalTok{\}}
    \NormalTok{];}
\end{Highlighting}
\end{Shaded}

Once this definition is in place, we can drop the specification of the
\texttt{as} option for each Question element. The final data file looks
like this:

\begin{Shaded}
\begin{Highlighting}[]
    \KeywordTok{var} \NormalTok{shuffleSequence = }\FunctionTok{seq}\NormalTok{(anyType);}

    \KeywordTok{var} \NormalTok{defaults = [}
        \StringTok{"Question"}\NormalTok{, \{}\DataTypeTok{as}\NormalTok{: [}\StringTok{"Yes"}\NormalTok{, }\StringTok{"No"}\NormalTok{]\},}
        \StringTok{"DashedSentence"}\NormalTok{, \{}\DataTypeTok{mode}\NormalTok{: }\StringTok{"speeded acceptability"}\NormalTok{\}}
    \NormalTok{];}

    \KeywordTok{var} \NormalTok{items = [}

    \NormalTok{[}\StringTok{"filler"}\NormalTok{, }\StringTok{"DashedSentence"}\NormalTok{, \{}\DataTypeTok{s}\NormalTok{: }\StringTok{"Here's a silly filler sentence"}\NormalTok{\},}
               \StringTok{"Question"}\NormalTok{, \{}\DataTypeTok{q}\NormalTok{: }\StringTok{"Is this a filler sentence?"}\NormalTok{\}],}
    \NormalTok{[}\StringTok{"filler"}\NormalTok{, }\StringTok{"DashedSentence"}\NormalTok{, \{}\DataTypeTok{s}\NormalTok{: }\StringTok{"And another silly filler sentence"}\NormalTok{\},}
               \StringTok{"Question"}\NormalTok{, \{}\DataTypeTok{q}\NormalTok{: }\StringTok{"Does this sentence have a relative clause?"}\NormalTok{\}],}
    \NormalTok{[}\StringTok{"relclause"}\NormalTok{, }\StringTok{"DashedSentence"}\NormalTok{, \{}\DataTypeTok{s}\NormalTok{: }\StringTok{"A sentence that has a relative clause"}\NormalTok{\},}
                  \StringTok{"Question"}\NormalTok{, \{}\DataTypeTok{q}\NormalTok{: }\StringTok{"Was there movement through [Spec,CP]?"}\NormalTok{\}],}
    \NormalTok{[}\StringTok{"relclause"}\NormalTok{, }\StringTok{"DashedSentence"}\NormalTok{, \{}\DataTypeTok{s}\NormalTok{: }\StringTok{"Another sentence that has a relative clause"}\NormalTok{\},}
                  \StringTok{"Question"}\NormalTok{, \{}\DataTypeTok{q}\NormalTok{: }\StringTok{"Was the first word of that sentence 'frog'?"}\NormalTok{\}]}

    \NormalTok{];}
\end{Highlighting}
\end{Shaded}

\textbf{Where to put your data file}: Data files live in the
\texttt{data\_includes} directory, and must have a \texttt{.js}
extension. You can only have data one file in the directory at any one
time, since if you have multiple files, they will just overwrite the
definitions for \texttt{shuffleSequence}, \texttt{items} and
\texttt{defaults}. However, you can tell the server to ignore some of
the files in \texttt{data\_includes} (see the section ``Configuring
\texttt{js\_includes}, \texttt{data\_includes} and
\texttt{css\_includes}'').

\section{Shuffle sequences}\label{shuffle-sequences}

\subsection{\texorpdfstring{\texttt{seq}, \texttt{randomize} and
\texttt{shuffle}}{seq, randomize and shuffle}}\label{seq-randomize-and-shuffle}

A ``shuffle sequence'' is a JavaScript data structure describing a
series of ``shuffling'', randomizing and sequencing operations over an
array of items. A variable called \texttt{shuffleSequence} should be
defined in your data file with a data structure of this sort as its
value.

Shuffle sequences are composed of three basic operations, \texttt{seq},
\texttt{randomize} and \texttt{shuffle}. Both take a series of ``type
predicates'' as arguments, where each type predicate is the
characteristic function of a set of types. A type predicate may be one
of the following:

\begin{itemize}
\item
  A string or integer, denoting the characteristic function of all items
  of the given type.
\item
  A JavaScript function, returning a boolean value when given a type
  (either a string or integer).
\item
  Another shuffle sequence (shuffle sequences can be embedded inside
  bigger shuffle sequences).
\end{itemize}

The basic operations work as follows:

\begin{itemize}
\item
  A statement of the form \texttt{seq(pred1,\ pred2,\ pred3,\ ...)}
  specifies a sequence where all items matching \texttt{pred1} precede
  all items matching \texttt{pred2}, all items matching \texttt{pred2}
  precede all items matching \texttt{pred3}, and so on. The original
  relative ordering between items of the same type is preserved. A
  \texttt{seq} with only one argument is permissible.
\item
  A statement of the form \texttt{randomize(pred1)} specifies a randomly
  ordered sequence of all items matching pred1.
\item
  A statement of the form \texttt{shuffle(pred1,\ pred2,\ pred3,\ ...)}
  specifies that items matching the given predicates should be shuffled
  together in such a way that items matching each predicate are evenly
  spaced. The original relative ordering between items of the same type
  is preserved.
\end{itemize}

The following type predicates are predefined as JavaScript functions:

\begin{longtable}[c]{p{1in}p{5in}}
\toprule
\textbf{Function} & \textbf{Description}\tabularnewline
\midrule
\endhead
\texttt{anyType} & Matches any type.\tabularnewline
\texttt{lessThan0} & Matches any integer type \textless{}
0.\tabularnewline
\texttt{greaterThan0} & Matches any integer type \textgreater{}
0.\tabularnewline
\texttt{equalTo0} & Matches any integer type = 0.\tabularnewline
\texttt{startsWith(s)} & Matches any string type starting with
s.\tabularnewline
\texttt{endsWith(s)} & Matches any string type ending with
s.\tabularnewline
\texttt{not(pred)} & Matches anything that is not of a type matched by
pred.\tabularnewline
\texttt{anyOf(p1,\ p2,\ ...)} & Takes any number of type predicates as
its arguments, and matches anything matching one of these
predicates.\tabularnewline
\bottomrule
\end{longtable}

If you define your own predicates, be careful to test that they are
cross-browser compatible. See the ``Cross-browser compatibility''
section below for some pertinent advice. The predicates above are
defined in \texttt{shuffle.js}.

The power of shuffle sequences derives from the possibility of composing
them without limit. Suppose we want the following order: all practice
items in their original order followed by evenly spaced real and filler
items in random order. Assuming the types ``practice'', ``real'' and
``filler'', we could use the following shuffle sequence:

\begin{Shaded}
\begin{Highlighting}[]
    \FunctionTok{seq}\NormalTok{(}\StringTok{"practice"}\NormalTok{, }\FunctionTok{shuffle}\NormalTok{(}\FunctionTok{randomize}\NormalTok{(}\StringTok{"real"}\NormalTok{), }\FunctionTok{randomize}\NormalTok{(}\StringTok{"filler"}\NormalTok{)))}
\end{Highlighting}
\end{Shaded}

Now suppose that there are two types of real item (``real1'' and
``real2''), and we wish to order the items the same way as before:

\begin{Shaded}
\begin{Highlighting}[]
    \FunctionTok{seq}\NormalTok{(}\StringTok{"practice"}\NormalTok{, }\FunctionTok{shuffle}\NormalTok{(}\FunctionTok{randomize}\NormalTok{(}\FunctionTok{anyOf}\NormalTok{(}\StringTok{"real1"}\NormalTok{, }\StringTok{"real2"}\NormalTok{)),}
                    \FunctionTok{randomize}\NormalTok{(}\StringTok{"filler"}\NormalTok{)))}
\end{Highlighting}
\end{Shaded}

What if we also want the two types of real item to be evenly spaced? The
following formula will do the trick:

\begin{Shaded}
\begin{Highlighting}[]
    \FunctionTok{seq}\NormalTok{(}\StringTok{"practice"}\NormalTok{, }\FunctionTok{shuffle}\NormalTok{(}\FunctionTok{randomize}\NormalTok{(}\StringTok{"filler"}\NormalTok{),}
                            \FunctionTok{shuffle}\NormalTok{(}\FunctionTok{randomize}\NormalTok{(}\StringTok{"real1"}\NormalTok{),}
                                    \FunctionTok{randomize}\NormalTok{(}\StringTok{"real2"}\NormalTok{))))}
\end{Highlighting}
\end{Shaded}

This first shuffles items of type ``real1'' and items of type ``real2''
and then shuffles filler items into the mix. Finally, practice items are
prepended in the order they were given in \texttt{data.js}.

Since it is often useful to apply \texttt{randomize} to every argument
of \texttt{shuffle}, there is a utility function, \texttt{rshuffle},
which automates this. The following equivalence holds:

\begin{Shaded}
\begin{Highlighting}[]
    \FunctionTok{rshuffle}\NormalTok{(a1, a2, ...) = }\FunctionTok{shuffle}\NormalTok{(}\FunctionTok{randomize}\NormalTok{(a1), }\FunctionTok{randomize}\NormalTok{(a2), ...)}
\end{Highlighting}
\end{Shaded}

If no shuffle sequence is specified, Ibex uses the following default
sequence:

\begin{Shaded}
\begin{Highlighting}[]
    \FunctionTok{seq}\NormalTok{(equalTo0, }\FunctionTok{rshuffle}\NormalTok{(greaterThan0, lessThan0))}
\end{Highlighting}
\end{Shaded}

This will seem rather cryptic to anyone not familiar with the behavior
of earlier versions of Ibex, where this ordering specification was built
in and unchangeable. In short, it works well if practice items have type
0, filler items have integer types \textless{} 0, and real items have
integer types \textgreater{} 0.

\textbf{Important}: A shuffle sequence must always have one of
\texttt{seq}, \texttt{shuffle}, \texttt{randomize} or \texttt{rshuffle}
as its outer element. A single string or integer is \emph{not} a valid
shuffle sequence, and neither is a predicate expression such as
\texttt{not("foo")}. Thus, one must use \texttt{seq("foo")}, not just
\texttt{"foo"}, and \texttt{seq(not("foo"))}, not just
\texttt{not("foo")}.

It is possible to include duplicate items in the final sequence. For
example, \texttt{seq("foo",\ "foo")} would include every item of type
``foo'' twice. For this reason, it is possible to accidentally include
duplicate items if a number of your predicates overlap. You can define
your own shuffle sequence operators and predicates quite easily; see
\texttt{shuffle.js} for the definitions of the \texttt{seq},
\texttt{shuffle} and \texttt{randomize} operators.

\subsection{Adding separators}\label{adding-separators}

Normally, it is a good idea to have some sort of padding in between
items to warn participants that they have finished one item and are
starting another. Ibex provides the \texttt{Separator} item for this
purpose. It can either work on a timeout or prompt for a keypress.
Here's an example:

\begin{Shaded}
\begin{Highlighting}[]
    \NormalTok{[}\StringTok{"sep"}\NormalTok{, }\StringTok{"Separator"}\NormalTok{, \{}\DataTypeTok{transfer}\NormalTok{: }\DecValTok{1000}\NormalTok{, }\DataTypeTok{normalMessage}\NormalTok{: }\StringTok{"Please wait for the next sentence."}\NormalTok{\}]}
\end{Highlighting}
\end{Shaded}

When the \texttt{transfer} option is set to 1000, this specifies that
there should be a 1000ms wait before the next item. The
\texttt{normalMessage} option gives the message that should be displayed
if the participant didn't do anything wrong on the previous item (see
the section ``Communication between elements'' for more details). If the
other items have failure conditions, you should set the
\texttt{errorMessage} option too. If you want the participant to proceed
by pressing a key rather than waiting, set \texttt{transfer} to
``keypress''.

The shuffle sequence operator \texttt{sepWith} is provided for the
purpose of interpolating separators with other items. It takes two
arguments: the first is a shuffle sequence specifying the sequence of
items which should be used to separate the other items; the second
argument is a shuffle sequence specifying the other items.

Let's take the example data file above and add timeout Separators
between each item. There are no failure conditions in this experiment
(since there is no ``wrong'' answer for an acceptability judgment) so we
only need to specify the \texttt{normalMessage} option. For the moment,
we'll still present the sentences in the order in which they appear in
the \texttt{items} array. Here's the modified file:

\begin{Shaded}
\begin{Highlighting}[]
    \KeywordTok{var} \NormalTok{shuffleSequence = }\FunctionTok{sepWith}\NormalTok{(}\StringTok{"sep"}\NormalTok{, }\FunctionTok{not}\NormalTok{(}\StringTok{"sep"}\NormalTok{));}

    \KeywordTok{var} \NormalTok{defaults = [}
        \StringTok{"Question"}\NormalTok{, \{}\DataTypeTok{as}\NormalTok{: [}\StringTok{"Yes"}\NormalTok{, }\StringTok{"No"}\NormalTok{]\},}
        \StringTok{"DashedSentence"}\NormalTok{, \{}\DataTypeTok{mode}\NormalTok{: }\StringTok{"speeded acceptability"}\NormalTok{\}}
    \NormalTok{];}

    \KeywordTok{var} \NormalTok{items = [}

    \NormalTok{[}\StringTok{"sep"}\NormalTok{, }\StringTok{"Separator"}\NormalTok{, \{}\DataTypeTok{transfer}\NormalTok{: }\DecValTok{1500}\NormalTok{, }\DataTypeTok{normalMessage}\NormalTok{: }\StringTok{"Please wait for the next item."}\NormalTok{\}],}

    \NormalTok{[}\StringTok{"filler"}\NormalTok{, }\StringTok{"DashedSentence"}\NormalTok{, \{}\DataTypeTok{s}\NormalTok{: }\StringTok{"Here's a silly filler sentence"}\NormalTok{\},}
               \StringTok{"Question"}\NormalTok{, \{}\DataTypeTok{q}\NormalTok{: }\StringTok{"Is this a filler sentence?"}\NormalTok{\}],}
    \NormalTok{[}\StringTok{"filler"}\NormalTok{, }\StringTok{"DashedSentence"}\NormalTok{, \{}\DataTypeTok{s}\NormalTok{: }\StringTok{"And another silly filler sentence"}\NormalTok{\},}
               \StringTok{"Question"}\NormalTok{, \{}\DataTypeTok{q}\NormalTok{: }\StringTok{"Does this sentence have a relative clause?"}\NormalTok{\}],}
    \NormalTok{[}\StringTok{"relclause"}\NormalTok{, }\StringTok{"DashedSentence"}\NormalTok{, \{}\DataTypeTok{s}\NormalTok{: }\StringTok{"A sentence that has a relative clause"}\NormalTok{\},}
                  \StringTok{"Question"}\NormalTok{, \{}\DataTypeTok{q}\NormalTok{: }\StringTok{"Was there movement through [Spec,CP]?"}\NormalTok{\}],}
    \NormalTok{[}\StringTok{"relclause"}\NormalTok{, }\StringTok{"DashedSentence"}\NormalTok{, \{}\DataTypeTok{s}\NormalTok{: }\StringTok{"Another sentence that has a relative clause"}\NormalTok{\},}
                  \StringTok{"Question"}\NormalTok{, \{}\DataTypeTok{q}\NormalTok{: }\StringTok{"Was the first word of that sentence 'frog'?"}\NormalTok{\}]}

    \NormalTok{];}
\end{Highlighting}
\end{Shaded}

\subsection{Manipulating individual
items}\label{manipulating-individual-items}

So far, we've been treating items as atoms for the purposes of shuffle
sequences -- although an item might be composed of several elements, the
operations \texttt{seq}, \texttt{randomize}, \texttt{shuffle} and
\texttt{sepWith} ignore this internal structure. However, it is
sometimes useful to be able to append or prepend a particular sequence
of elements to every item. For example, if you are doing a speeded
acceptability judgment task, you might want every sentence to be
followed by exactly the same question (``Was this a good sentence?'').
To this end, webspr provides the \texttt{precedeEachWith} and
\texttt{followEachWith} operations.

Both functions take two arguments, each of which should be a shuffle
sequence. The first argument specifies the sequence of items which (when
flattened to a sequence of elements) is to be appended/prepended to
every item in the second argument. Returning to the previous example
data file, let's modify it so that every sentence is followed by the
same acceptability question. We'll also ensure that the sentences are
randomly ordered, with the fillers and ``real'' sentences evenly spaced:

\begin{Shaded}
\begin{Highlighting}[]
    \KeywordTok{var} \NormalTok{shuffleSequence = }\FunctionTok{followEachWith}\NormalTok{(}\StringTok{"question"}\NormalTok{, }\FunctionTok{rshuffle}\NormalTok{(}\StringTok{"filler"}\NormalTok{, }\StringTok{"relclause"}\NormalTok{));}

    \KeywordTok{var} \NormalTok{defaults = [}
        \StringTok{"DashedSentence"}\NormalTok{, \{}\DataTypeTok{mode}\NormalTok{: }\StringTok{"speeded acceptability"}\NormalTok{\}}
    \NormalTok{];}

    \KeywordTok{var} \NormalTok{items = [}

    \NormalTok{[}\StringTok{"question"}\NormalTok{, }\StringTok{"Question"}\NormalTok{, \{}\DataTypeTok{q}\NormalTok{: }\StringTok{"Was that a good sentence?"}\NormalTok{, }\DataTypeTok{as}\NormalTok{: [}\StringTok{"Yes"}\NormalTok{, }\StringTok{"No"}\NormalTok{]\}],}

    \NormalTok{[}\StringTok{"filler"}\NormalTok{, }\StringTok{"DashedSentence"}\NormalTok{, \{}\DataTypeTok{s}\NormalTok{: }\StringTok{"Here's a silly filler sentence"}\NormalTok{\}],}
    \NormalTok{[}\StringTok{"filler"}\NormalTok{, }\StringTok{"DashedSentence"}\NormalTok{, \{}\DataTypeTok{s}\NormalTok{: }\StringTok{"And another silly filler sentence"}\NormalTok{\}],}

    \NormalTok{[}\StringTok{"relclause"}\NormalTok{, }\StringTok{"DashedSentence"}\NormalTok{, \{}\DataTypeTok{s}\NormalTok{: }\StringTok{"A sentence that has a relative clause"}\NormalTok{\}],}
    \NormalTok{[}\StringTok{"relclause"}\NormalTok{, }\StringTok{"DashedSentence"}\NormalTok{, \{}\DataTypeTok{s}\NormalTok{: }\StringTok{"Another sentence that has a relative clause"}\NormalTok{\}]}

    \NormalTok{];}
\end{Highlighting}
\end{Shaded}

\textbf{Important:} \texttt{precedeEachWith} and \texttt{followEachWith}
have a somewhat confusing behavior with respect to the format of the
results file. Although in effect, the first argument of
\texttt{followEachWith} is appended to every item, the appended elements
are not considered a part of the items they are appended to in the
results file. Rather, every question in the preceding example will have
the same item number and element number in the results file (that based
on the position of the question item in the data file).

\subsection{Latin square designs}\label{latin-square-designs}

Ibex has built-in support for latin square designs. These are
implemented by assigning each item a \emph{group} in addition to its
type. Each participant sees only one item out of all the items in any
given group. To give an example, we could place both of the
``relclause'' sentences from the previous example in the same group
using the following code:

\begin{Shaded}
\begin{Highlighting}[]
    \NormalTok{[[}\StringTok{"relclause"}\NormalTok{, }\DecValTok{1}\NormalTok{], }\StringTok{"DashedSentence"}\NormalTok{, \{}\DataTypeTok{s}\NormalTok{: }\StringTok{"A sentence that has a relative clause"}\NormalTok{\}],}
    \NormalTok{[[}\StringTok{"relclause"}\NormalTok{, }\DecValTok{1}\NormalTok{], }\StringTok{"DashedSentence"}\NormalTok{, \{}\DataTypeTok{s}\NormalTok{: }\StringTok{"Another sentence that has a relative clause"}\NormalTok{\}]}
\end{Highlighting}
\end{Shaded}

Now, any given participant will see only one of these sentences. Designs
are rotated using a counter stored on the server. Groups, like types,
may either be strings or numbers. By default, webspr does not check that
all groups contain the same number of items (this behavior is new in
0.2.4). You can set the \texttt{equalGroupSizes} variable to
\texttt{true} in order to revert to the old behavior, where an error is
raised if groups contain differing number of items. (See the
``Miscellaneous options'' subsection below.)

\textbf{New:} You may now choose which latin square a participant will
be placed in by using a URL of the following form:
``\ldots{}/server.py?withsquare=XXXX''. This will set the latin square
to XXXX for the current participant and then display the main experiment
page. Selecting the latin square in this manner does \textbf{not} modify
the master counter stored on the server.

\textbf{Newer:} You can also set the counter in the data file, with a
statement such as \texttt{var\ counterOverride\ =\ 18;}. Again, the
server's latin square counter is not modified. This method takes
priority over the URL method described in the previous paragraph, if
both are used.

A new (and not very well-tested) feature allows for more complex designs
where selection of an item from one group is dependent on selection of
an item from another. For example, you can specify that the item chosen
from group 3 should be the same as the item chosen from group 2 (i.e.~if
the first item is chosen from group 2, the first item will also be
chosen from group 3). To use this feature, just replace each group
specifier with a pair of group specifiers, where the first member of the
pair is the original group specifier, and the second is the group which
governs selection from the original group. For example, suppose that we
have another group (group 2), and we want choices from this group to be
linked to choices from group 1 in the example above. The following code
will do the trick:

\begin{Shaded}
\begin{Highlighting}[]
    \NormalTok{[[}\StringTok{"relclause"}\NormalTok{, [}\DecValTok{2}\NormalTok{, }\DecValTok{1}\NormalTok{]], }\StringTok{"DashedSentence"}\NormalTok{,}
      \NormalTok{\{}\DataTypeTok{s}\NormalTok{: }\StringTok{"I am paired with 'A sentence that has a relative clause'"}\NormalTok{\}],}
    \NormalTok{[[}\StringTok{"relclause"}\NormalTok{, [}\DecValTok{2}\NormalTok{, }\DecValTok{1}\NormalTok{]], }\StringTok{"DashedSentence"}\NormalTok{,}
      \NormalTok{\{}\DataTypeTok{s}\NormalTok{: }\StringTok{"I am paired with 'Another sentence that has a relative clause'"}\NormalTok{\}]}
\end{Highlighting}
\end{Shaded}

Note that the second number in the pair must be the same for all items
in the group. \textbf{Warning:} This feature may be removed in future
releases if I get around to implementing a more general way of
implementing more complex latin square designs.

\subsection{Sending results early}\label{sending-results-early}

You can now control the point in the experiment at which Ibex sends the
results to the server. This allows you to (e.g.) present the participant
with a link to another website after the results have been successfully
uploaded. In order to use this feature, you must first set the
\texttt{manualSendResults} config variable to \texttt{true}. Then, add a
\texttt{\_\_SendResults\_\_} controller to your items list and insert it
somewhere in your shuffle sequence. There is some example code in
\texttt{data\_includes/example\_data.js}. Note that you may only send
results once per experiment -- it is not possible to incrementally
upload results.

If you set \texttt{manualSendResults} to \texttt{true}, but do not add a
\texttt{\_\_SendResults\_\_} controller to your shuffle sequence, then
the results will never get sent at all.

\subsection{Modifying the running order
manually}\label{modifying-the-running-order-manually}

Occasionally, shuffle sequences aren't powerful enough to arrange items
in the order you wish. To perform abritrary rearrangements of items, you
can define a \texttt{modifyRunningOrder} function in your data file.
This function takes as its input the running order generated by the
shuffleSequence, and returns a new runningOrder. (The function is
permitted to modify its argument, but it must return a running order.) A
running order is an array of arrays of elements. Each element is an
object with the following properties:

\begin{itemize}
\itemsep1pt\parskip0pt\parsep0pt
\item
  \texttt{itemNumber}
\item
  \texttt{elementNumber}
\item
  \texttt{type}
\item
  \texttt{group}
\item
  \texttt{controller} (a string giving the name of the controller)
\item
  \texttt{options} (an object giving the options for the controller)
\item
  \texttt{hideResults} (boolean; if true, results from this controller
  are not included in the results file)
\end{itemize}

You can create new objects with these properties and insert them into
the running order. Typically, these objects would have
\texttt{itemNumber}, \texttt{elementNumber}, \texttt{type} and
\texttt{group} set to \texttt{null}. In this case, if the controller
adds lines to the results file, then the ``item'', ``element'', ``type''
and ``group'' columns will all have the value ``DYNAMIC''.

The convenince constructor \texttt{DynamicElement} is provided to ease
the construction of typical element objects. This takes as its first
argument a controller name, and as its second the controller's options.
Optionally, \texttt{true} may be passed as the third argument to set the
\texttt{hideResults} property to \texttt{true}. The following example
inserts a ``pause'' \texttt{Message} at every tenth item:

\begin{Shaded}
\begin{Highlighting}[]
    \KeywordTok{function} \FunctionTok{modifyRunningOrder}\NormalTok{(ro) \{}
        \KeywordTok{for} \NormalTok{(}\KeywordTok{var} \NormalTok{i = }\DecValTok{0}\NormalTok{; i < }\OtherTok{ro}\NormalTok{.}\FunctionTok{length}\NormalTok{; ++i) \{}
            \KeywordTok{if} \NormalTok{(i % }\DecValTok{10} \NormalTok{== }\DecValTok{0}\NormalTok{) \{}
                \CommentTok{// Passing 'true' as the third argument casues the results from this controller}
                \CommentTok{// to be omitted from the results file. (Though in fact, the Message controller}
                \CommentTok{// does not add any results in any case.)}
                \NormalTok{ro[i].}\FunctionTok{push}\NormalTok{(}\KeywordTok{new} \FunctionTok{DynamicElement}\NormalTok{(}
                    \StringTok{"Message"}\NormalTok{,}
                    \NormalTok{\{ }\DataTypeTok{html}\NormalTok{: }\StringTok{"<p>Pause</p>"}\NormalTok{, }\DataTypeTok{transfer}\NormalTok{: }\DecValTok{1000} \NormalTok{\},}
                    \KeywordTok{true}
                \NormalTok{));}
            \NormalTok{\}}
        \NormalTok{\}}
        \KeywordTok{return} \NormalTok{ro;}
    \NormalTok{\}}
\end{Highlighting}
\end{Shaded}

\section{Communication between
elements}\label{communication-between-elements}

Ibex allows for a limited amount of communication between elements. Each
element may set keys in a hashtable which is passed to the next element.
The hashtable is cleared between elements, so there is no possibility of
long-distance communication.

Currently, this system is used to provide feedback to participants when
they (for example) answer a comprehension question incorrectly.
Controllers such as \texttt{Question} set the key ``failed'' if
something goes wrong, and the next \texttt{Separator} item is then able
to display a message warning the participant that they answered
incorrectly.

\section{Controllers}\label{controllers}

For all controllers, the \texttt{hideProgressBar} option may be set to
true, in order to prevent the progress bar being displayed while the
controller is in action. Similarly, the \texttt{countsForProgressBar}
option may be set to determine whether or not completion of the
controller causes the indicator in the progress bar to move to the
right. (Different controllers have a different default setting for
\texttt{countsForProgressBar}.)

\subsection{Separator}\label{separator}

\textbf{Options}

\begin{longtable}[c]{p{1in}p{1in}p{3.85in}}
\toprule
\textbf{Option} & \textbf{Default} & \textbf{Description}\tabularnewline
\midrule
\endhead
transfer & \texttt{"keypress"} & Must be either \texttt{"keypress"} or a
number. If the former, participant goes to the next item by pressing any
key. If the latter, specifies number of ms to wait before the next
item.\tabularnewline
normalMessage & \texttt{"Press\ any\ key\ to\ continue."} & Message to
display if the previous item was completed normally.\tabularnewline
errorMessage & \texttt{"Wrong.\ Press\ any\ key\ to\ continue."} &
Message to display (in red) if the previous item was not completed
normally (e.g.~timeout, incorrect answer).\tabularnewline
ignoreFailure & \texttt{false} & If true, never displays an error
message.\tabularnewline
\bottomrule
\end{longtable}

\textbf{Results}

\texttt{Separator} does not add any lines to the results file.

\subsection{Message}\label{message}

\textbf{Options}

\begin{longtable}[c]{p{1in}p{1in}p{3.85in}}
\toprule
\textbf{Option} & \textbf{Default} & \textbf{Description}\tabularnewline
\midrule
\endhead
html & \emph{obligatory} & The HTML for the message to display (see
section ``HTML Code'' below).\tabularnewline
transfer & \texttt{"click"} & Either \texttt{"click"},
\texttt{"keypress}``, an integer, or \texttt{null}. If \texttt{"click"},
the participant clicks a link at the bottom of the message to continue
(see''continueMessage" option). If \texttt{"keypress"}, they press any
key to continue. If an integer, the experiment continues after pausing
for the specified number of milliseconds. If \texttt{null}, there is no
way for the user to complete the controller (and no ``click to
continue'' message is displayed). This is useful only when the
controller is part of a larger \texttt{VBox}.\tabularnewline
consentRequired & \texttt{false} & If true, the participant is required
to tick a checkbox indicating that they consent to do the experiment (in
this case, the message would probably be some sort of statement of
terms/conditions). This option can only be set to \texttt{true} if the
``transfer'' option is set go \texttt{"click"}. Note that it is also
possible to create checkboxes like this using the more flexible
\texttt{Form} controller.\tabularnewline
continueMessage & \texttt{"..."} & Only valid if the ``consentRequired''
option is set to \texttt{"true"}. This specifies the text that will
appear in the link that the participant needs to click to continue with
the experiment.\tabularnewline
consentMessage & \texttt{"..."} & Only valid if the ``consentRequired''
option is set to \texttt{"true"}. This specifies the text that will
appear next to the checkbox.\tabularnewline
consentErrorMessage & \texttt{"..."} & Only valid if the
``consentRequired'' option is set to \texttt{"true"}. This specifies the
error message that will be given if the participant attempts to continue
without checking the consent checkbox.\tabularnewline
\bottomrule
\end{longtable}

\textbf{Results}

\texttt{Message} does not add any lines to the results file.

\subsection{DashedSentence}\label{dashedsentence}

\textbf{Options}

\begin{longtable}[c]{p{1in}p{1in}p{3.85in}}
\toprule
\textbf{Option} & \textbf{Default} & \textbf{Description}\tabularnewline
\midrule
\endhead
s & \emph{obligatory} & The sentence. This is either a string, in which
case the sentence will be presented word-by-word, or a list of strings
(``chunks''), in which case the sentence will be presented
chunk-by-chunk. If one of the words/chunks begins with the character
\href{mailto:"@"}{\nolinkurl{"@"}}, then the controller will finish
after the word beginning with ``@'' is displayed (the ``@'' will be
stripped when the word is presented). This feature can be useful if you
want (for example) to interrupt a self-paced reading item with a lexical
decision task.\tabularnewline
mode & \texttt{"self-paced\ reading"} & Either
\texttt{"self-paced\ reading"} or
\texttt{"speeded\ acceptability"}.\tabularnewline
display & \texttt{"dashed"} & If set to \texttt{"dashed"}, the sentence
is displayed as a sequence of dashes; when the participant presses the
space bar, the current word is displayed above the corresponding dash,
and no other words are displayed. If set to \texttt{"in\ place"}, words
are displayed one-by-one in the center of the screen. Setting to
\texttt{"in\ place"} changes the default value of \texttt{wordPauseTime}
from 100 to 0 and the default value of \texttt{wordTime} from 300 to
400, since otherwise the display flickers.\tabularnewline
blankText & \texttt{"\textbackslash{}u2014\textbackslash{}u2014"} &
Applicable only if ``display'' is set to \texttt{"in\ place"}. This is
the text that is shown before the participant presses space for the
first time. By default it is two horizontal dashes.\tabularnewline
wordTime & \texttt{300} & If mode is \texttt{"speeded\ acceptability"},
the time in ms each word should be displayed for. (See also description
of ``display'' option.)\tabularnewline
wordPauseTime & \texttt{100} & If mode is
\texttt{"speeded\ acceptability"}, the time in ms a word should remain
blank before it is shown. (See also description of ``display''
option.)\tabularnewline
sentenceDescType & \texttt{"literal"} & Determines the format of column
1 of the results (see table below). If \texttt{"literal"}, column
contains the sentence itself (encoded as a URL with \%XX escapes).
Currently, this is the only possible value for this
option.\tabularnewline
showAhead & \texttt{true} & Whether or not to use dashes to indicate the
presence of words ahead of the reader's current position.\tabularnewline
showBehind & \texttt{true} & As above (but behind rather than ahead). If
both showAhead and showBehind are set to \texttt{false}, no dashes are
shown.\tabularnewline
hideUnderscores & \texttt{false} & If this is set to \texttt{true}, the
underscore character is interpreted as a space which does not trigger a
break between words. This is an alternative to passing an array of
strings as the value of the \texttt{s} option.\tabularnewline
\bottomrule
\end{longtable}

\textbf{Note on right-to-left languages:} The current version of Ibex
doesn't provide out-of-the-box support for doing self-paced reading
experiments with languages that are displayed right to left (although I
hope to add support for this soon). However, it is possible to modify
the code for the \texttt{DashedSentence} controller to display in
right-to-left order. Please contact me
(\href{mailto:a.d.drummond@gmail.com}{\nolinkurl{a.d.drummond@gmail.com}})
if you want to do this.

\textbf{Results}

The format of the results depends on the setting of the \texttt{mode}
option. If it is set to ``speeded acceptability'', results have the
following format:

\begin{longtable}[c]{p{1in}p{5in}}
\toprule
\textbf{Column} & \textbf{Description}\tabularnewline
\midrule
\endhead
1 & See documentation for `sentenceDescType' option
above.\tabularnewline
\bottomrule
\end{longtable}

If \texttt{mode} is set to ``self-paced reading'', the results look like
this:

\begin{longtable}[c]{p{1in}p{5in}}
\toprule
\textbf{Column} & \textbf{Description}\tabularnewline
\midrule
\endhead
1 & Word number. For example, ``1'' indicates that this line gives the
time it took to read the first word (i.e.~the difference between and
).\tabularnewline
2 & The word.\tabularnewline
3 & The reading time in ms.\tabularnewline
4 & Either 1 or 0. Indicates whether or not there was a line break
between word n and word n + 1 (where reading time is for word
n).\tabularnewline
5 & See documentation for `sentenceDescType' option
above.\tabularnewline
\bottomrule
\end{longtable}

\subsection{FlashSentence}\label{flashsentence}

\textbf{Options}

\begin{longtable}[c]{p{1in}p{1in}p{3.85in}}
\toprule
\textbf{Option} & \textbf{Default} & \textbf{Description}\tabularnewline
\midrule
\endhead
s & \emph{obligatory} & The sentence to be displayed.\tabularnewline
timeout & 2000 & If \texttt{null}, the sentence is displayed
indefinitely (only useful if part of a VBox). Otherwise, a number giving
the time in ms to display the sentence.\tabularnewline
sentenceDescType & \texttt{"md5"} & See documentation for DashedSentence
controller.\tabularnewline
\bottomrule
\end{longtable}

\textbf{Results}

\begin{longtable}[c]{p{1in}p{5in}}
\toprule
\textbf{Column} & \textbf{Description}\tabularnewline
\midrule
\endhead
1 & See documentation for `sentenceDescType' option of the
DashedSentence controller.\tabularnewline
\bottomrule
\end{longtable}

\subsection{Question}\label{question}

\textbf{Options}

\begin{longtable}[c]{p{1in}p{1in}p{3.85in}}
\toprule
\textbf{Option} & \textbf{Default} & \textbf{Description}\tabularnewline
\midrule
\endhead
q & \texttt{null} & The question to pose.\tabularnewline
as & \emph{obligatory} & A list of strings giving the answers the user
has to choose from. May also be a list of pairs of strings (e.g.
\texttt{{[}"y",\ "Yes"{]}}. In this case, the second string gives the
answer, and the first string specifies a key which the user may press to
select that answer. Note that these key assignments will be ignored if
\texttt{presentAsScale} is \texttt{true} (but they will not be ignored
if \texttt{presentHorizontally} is true).\tabularnewline
instructions & \texttt{null} & Instructions for answering the question,
displayed following the question and the answer
selection.\tabularnewline
hasCorrect & \texttt{false} & If \texttt{false}, indicates that none of
the answers is privileged as correct. Otherwise, is either
\texttt{true}, indicating that the first answer in the \texttt{as} list
is the correct one; an integer, giving the index of the correct answer
in the \texttt{as} list (starting from 0); or a string, giving the
correct answer.\tabularnewline
autoFirstChar & \texttt{false} & If set to true, users can select
answers by pressing the key corresponding to the first letter in the
answer. (This is useful e.g.~for yes/no questions.)\tabularnewline
showNumbers & \texttt{true} & If \texttt{true}, answers are numbered and
participants can use number keys to select them.\tabularnewline
randomOrder & \texttt{true} if the question has a correct answer, false
otherwise. & Whether or not to randomize the order of the answers before
displaying them. If \texttt{presentAsScale} or
\texttt{presentHorizontally} are set to \texttt{true},
\texttt{randomOrder} can also be set to a list of keys. This list should
be the same length as \texttt{as}. It specifies keys in a left-to-right
order. This makes it possible for the mapping from keys to answers to
correspond to the (random) order in which the answers are
presented.\tabularnewline
presentAsScale & \texttt{false} & If \texttt{true}, answers are
presented as a scale (useful for acceptability ratings). If any of the
points in the scale are integers 0-9, the participant may select them by
pressing a number key.\tabularnewline
presentHorizontally & \texttt{false} & Note that at most one of
\texttt{presentAsScale} and \texttt{presentHorizontally} should be set
to \texttt{true}. \texttt{presentHorizontally} is like
\texttt{presentAsScale} except that it does not override the usual means
of specifying which key presses trigger which answers, and it leaves
\texttt{randomOrder} on by default for questions which have correct
answers. If your scale contains words, you may use the
\texttt{autoFirstChar} option, and in all cases, you may explicitly
specify answer keys, as described in the comment on the \texttt{as}
option.\tabularnewline
leftComment & \texttt{null} & \emph{Only valid if
\texttt{presentAsScale} is \texttt{true}.} This option specifies text to
be displayed at the left edge of a scale (e.g. ``Bad'').\tabularnewline
rightComment & \texttt{null} & As for ``leftComment'', but for the right
edge of the scale.\tabularnewline
timeout & \texttt{null} & If \texttt{null}, there is no time limit.
Otherwise, should be a number giving the time in ms a participant has to
answer the question.\tabularnewline
\bottomrule
\end{longtable}

\textbf{Results}

\begin{longtable}[c]{p{1in}p{5in}}
\toprule
\textbf{Column} & \textbf{Description}\tabularnewline
\midrule
\endhead
1 & The question that was posed (encoded as a URL with \%XX
escapes).\tabularnewline
2 & The answer that the participant gave (also encoded as a
URL).\tabularnewline
3 & ``NULL'' if the question as no correct answer. Otherwise, 1 if the
participant answered correctly, or 0 if they didn't.\tabularnewline
4 & The time the participant took to answer the question
(ms).\tabularnewline
\bottomrule
\end{longtable}

\subsection{AcceptabilityJudgment}\label{acceptabilityjudgment}

The \texttt{AcceptabilityJudgment} controller makes it straightforward
to present a sentence together with a rating scale. It is a combination
of the \texttt{FlashSentence} and \texttt{Message} controllers.

\begin{longtable}[c]{p{1in}p{1in}p{3.85in}}
\toprule
\textbf{Option} & \textbf{Default} & \textbf{Description}\tabularnewline
\midrule
\endhead
s & \emph{obligatory} & The sentence.\tabularnewline
q & \emph{obligatory} & The question.\tabularnewline
as & \emph{obligatory} & Answers (as for
\texttt{Question}).\tabularnewline
instructions & \texttt{null} & As for \texttt{Question}\tabularnewline
hasCorrect & \texttt{false} & As for \texttt{Question}\tabularnewline
autoFirstChar & \texttt{false} & As for \texttt{Question}\tabularnewline
showNumbers & \texttt{true} & As for \texttt{Question}\tabularnewline
randomOrder & \emph{as for \texttt{Question}} & As for
\texttt{Question}\tabularnewline
presentAsScale & \texttt{false} & As for
\texttt{Question}\tabularnewline
leftComment & \texttt{null} & As for \texttt{Question}\tabularnewline
rightComment & \texttt{null} & As for \texttt{Question}\tabularnewline
timeout & \texttt{null} & As for \texttt{Question}\tabularnewline
\bottomrule
\end{longtable}

\textbf{Results}

Each \texttt{AcceptabilityJudgment} adds \textbf{two} lines to the
results file. The first line is the same as for the
\texttt{FlashSentence} controller; the second line is the same as for
the \texttt{Question} controller.

\subsection{DashedAcceptabilityJudgment}\label{dashedacceptabilityjudgment}

This is like \texttt{AcceptabilityJudgment}, but it uses
\texttt{DashedSentence} instead of \texttt{FlashSentence}. This is
useful for running speeded acceptability judgment tasks (which otherwise
require some rather creative use of shuffle sequences).

\begin{longtable}[c]{p{1in}p{1in}p{3.85in}}
\toprule
\textbf{Option} & \textbf{Default} & \textbf{Description}\tabularnewline
\midrule
\endhead
s & \emph{obligatory} & The sentence.\tabularnewline
q & \emph{obligatory} & The question.\tabularnewline
as & \emph{obligatory} & Answers (as for
\texttt{Question}).\tabularnewline
instructions & \texttt{null} & As for \texttt{Question}\tabularnewline
hasCorrect & \texttt{false} & As for \texttt{Question}\tabularnewline
autoFirstChar & \texttt{false} & As for \texttt{Question}\tabularnewline
showNumbers & \texttt{true} & As for \texttt{Question}\tabularnewline
randomOrder & \emph{as for \texttt{Question}} & As for
\texttt{Question}\tabularnewline
presentAsScale & \texttt{false} & As for
\texttt{Question}\tabularnewline
leftComment & \texttt{null} & As for \texttt{Question}\tabularnewline
rightComment & \texttt{null} & As for \texttt{Question}\tabularnewline
timeout & \texttt{null} & As for \texttt{Question}\tabularnewline
mode & \texttt{"speeded\ acceptability"} & As for
\texttt{DashedSentence} (but with different default)\tabularnewline
display & \texttt{"dashed"} & As for
\texttt{DashedSentence}\tabularnewline
blankText & \texttt{"\textbackslash{}u2014\textbackslash{}u2014"} & As
for \texttt{DahsedSentence}\tabularnewline
wordTime & \texttt{300} & As for \texttt{DashedSentence}\tabularnewline
wordPauseTime & \texttt{100} & As for
\texttt{DashedSentence}\tabularnewline
sentenceDescType & \texttt{"literal"} & As for
\texttt{DashedSentence}\tabularnewline
showAhead & \texttt{true} & As for
\texttt{DashedSentence}\tabularnewline
showBehind & \texttt{true} & As for
\texttt{DashedSentence}\tabularnewline
\bottomrule
\end{longtable}

\textbf{Results}

Each \texttt{DashedAcceptabilityJudgment} adds \textbf{two} lines to the
results file if the ``mode'' option is set to ``speeded acceptability''.
If it is set to ``self-paced reading'', the number of lines is 1 + the
number of words in the sentence. The first line(s) are the same as for
the \texttt{DashedSentence} controller; the last line is the same as for
the \texttt{Question} controller.

\subsection{VBox}\label{vbox}

The \texttt{VBox} controller makes it possible to combine multiple
controllers to form a single controller. Each controller in the VBox is
displayed at the same time, one on top of the other. This allows the
functionality of simple controllers to be reused in the construction of
more complex controllers. For an example of a controller constructed
using a VBox, see \texttt{js\_includes/acceptability\_judgment.js}.

Currently, you canno create VBox controllers inline in your data file.
Instead, you must create a new controller which passes its
\texttt{options} dictionary to the VBox. E.g., by placing something like
the following at the beginning of your data file:

\begin{Shaded}
\begin{Highlighting}[]
\FunctionTok{define_ibex_controller}\NormalTok{(\{}
    \DataTypeTok{name}\NormalTok{: }\StringTok{"MyController"}\NormalTok{,}

    \DataTypeTok{jqueryWidget}\NormalTok{: \{}
        \DataTypeTok{_init}\NormalTok{: }\KeywordTok{function} \NormalTok{() \{}
            \KeywordTok{this}\NormalTok{.}\OtherTok{options}\NormalTok{.}\FunctionTok{transfer} \NormalTok{= }\KeywordTok{null}\NormalTok{; }\CommentTok{// Remove 'click to continue message'.}
            \KeywordTok{this}\NormalTok{.}\OtherTok{element}\NormalTok{.}\FunctionTok{VBox}\NormalTok{(\{}
                \DataTypeTok{options}\NormalTok{: }\KeywordTok{this}\NormalTok{.}\FunctionTok{options}\NormalTok{,}
                \DataTypeTok{triggers}\NormalTok{: [}\DecValTok{1}\NormalTok{],}
                \DataTypeTok{children}\NormalTok{: [}
                    \StringTok{"Message"}\NormalTok{, }\KeywordTok{this}\NormalTok{.}\FunctionTok{options}\NormalTok{,}
                    \StringTok{"AcceptabilityJudgment"}\NormalTok{, }\KeywordTok{this}\NormalTok{.}\FunctionTok{options}\NormalTok{,}
                \NormalTok{]}
            \NormalTok{\});}
        \NormalTok{\}}
    \NormalTok{\},}

    \DataTypeTok{properties}\NormalTok{: \{ \}}
\NormalTok{\});}
\end{Highlighting}
\end{Shaded}

\textbf{Options}

\begin{longtable}[c]{p{1in}p{1in}p{3.85in}}
\toprule
\textbf{Option} & \textbf{Default} & \textbf{Description}\tabularnewline
\midrule
\endhead
children & \emph{obligatory} & An array of child controllers. Has
exactly the same format as an array of elements for an
item\tabularnewline
triggers & \emph{obligatory} & In order to determine when a VBox element
is complete, you must give an array of the indices of those of its
children which are ``triggers'' (indices start from 0). When each of the
trigger elements is complete, the VBox is complete.\tabularnewline
padding & \texttt{"2em"} & The amount of vertical padding to place
between the children. This should be a CSS dimension.\tabularnewline
\emph{vboxCallbackWhenChildFinishes} & \texttt{null} & If set, this
function is called every time one of the VBox's children finishes. It is
called with two paramaters: the index of the child, and an array of
arrays (= array of lines) giving the results for that
child.\tabularnewline
\bottomrule
\end{longtable}

\textbf{Results}

The \texttt{VBox} controller simply concatenates the results of its
children in the order that the children were given.

\subsection{Form}\label{form}

The \texttt{Form} controller collects data from an HTML form presented
to the participant. The form may contain any combination of text boxes
(\texttt{\textless{}input\ type="text"\textgreater{}}), text areas
(\texttt{\textless{}textarea\textgreater{}}), checkboxes
(\texttt{\textless{}input\ type="checkbox"\textgreater{}}) or radio
buttons (\texttt{\textless{}input\ type="radio"\textgreater{}}). It
isn't necessary to wrap the \texttt{\textless{}input\textgreater{}} tags
within a \texttt{\textless{}form\textgreater{}} tag; nor should you
include a submit button.

You can indicate that a text area or radio button group is obligatory by
adding the ``obligatory'' class to it (e.g.
\texttt{\textless{}input\ type="text"\ class="obligatory"\textgreater{}}).
For groups of radio buttons, you need only add the ``obligatory'' class
to one button in the group. (As usual, radio buttons in a single group
should all have the same `name' attribute.)

The ``obligatory'' class can also be used to indicate that a checkbox
must be checked before the user can continue. (This is useful for
consent checkboxes.)

By default, failure to fill in an obligatory field is signaled by an
alert dialog. However, you can indicate that the error message is to be
displayed on the page by inserting an empty label tag in your form:

\begin{Shaded}
\begin{Highlighting}[]
    \KeywordTok{<label}\OtherTok{ class=}\StringTok{"error"}\OtherTok{ for=}\StringTok{"FIELD_NAME"}\KeywordTok{></label>}
\end{Highlighting}
\end{Shaded}

If you want all error messages to be displayed in the same location, add
a \texttt{\textless{}label\textgreater{}} with `for' set to
``ALL\_FIELDS''.

\textbf{Options}

\begin{longtable}[c]{p{1in}p{1in}p{3.85in}}
\toprule
\textbf{Option} & \textbf{Default} & \textbf{Description}\tabularnewline
\midrule
\endhead
html & \emph{obligatory} & As for the \texttt{Message}
controller.\tabularnewline
continueOnReturn & \texttt{false} & If \texttt{true}, the user can
complete the form by pressing the return key when a single-line text
field is focused.\tabularnewline
continueMessage & ``Click here to continue'' & If set to \texttt{null},
no continue message is displayed.\tabularnewline
checkedValue & \texttt{"yes"} & The value stored in the results file
when a checkbox is checked.\tabularnewline
uncheckedValue & \texttt{"no"} & The value stored in the results file
when a checkbox is unchecked.\tabularnewline
validators & \texttt{\{\ \}} & A JavaScript dictionary mapping field
names to validating functions. Functions take the field value as their
sole argument, and return \texttt{true} if the value is valid or an
error string otherwise.\tabularnewline
saveReactionTime & \texttt{false} & If \texttt{true}, controller saves
the time elapsed between the form's being displayed and the
participant's completing it. Reaction time appears in the results file
as the value of a special \texttt{\_REACTION\_TIME\_}
field.\tabularnewline
errorCSSClass & error & The CSS class used to indicate that a
\texttt{\textless{}label\textgreater{}} tag is a placeholder for error
messages.\tabularnewline
obligatoryErrorGenerator & \texttt{...} & A function from field names to
error messages (used when an obligatory text box or text area is not
filled in).\tabularnewline
obligatoryCheckboxErrorGenerator & \texttt{...} & A function from field
names to error messages (used when an obligatory checkbox is not
checked).\tabularnewline
obligatoryRadioErrorGenerator & \texttt{...} & As above but for radio
button groups.\tabularnewline
\bottomrule
\end{longtable}

\textbf{Results}

For each \texttt{\textless{}input\textgreater{}} or
\texttt{\textless{}textarea\textgreater{}} tag, there is a line in the
results file with the following two columns:

\begin{longtable}[c]{p{1in}p{5in}}
\toprule
\textbf{Column} & \textbf{Description}\tabularnewline
\midrule
\endhead
1 & The name of the field (taken from the `name' attribute of the
\texttt{\textless{}input\textgreater{}} or
\texttt{\textless{}textarea\textgreater{}}).\tabularnewline
2 & The answer given by the participant.\tabularnewline
\bottomrule
\end{longtable}

\section{Further configuration}\label{further-configuration}

\subsection{Miscellaneous options}\label{miscellaneous-options}

You can set the values of the following variables in your data file
(e.g. \texttt{var\ showProgressBar\ =\ true;}).

\begin{longtable}[c]{p{1in}p{1in}p{3.85in}}
\toprule
\textbf{Option} & \textbf{Default} & \textbf{Description}\tabularnewline
\midrule
\endhead
showProgressBar & \texttt{true} & Whether or not to show a progress
bar.\tabularnewline
progressBarText & \texttt{progress} & The text that appears below the
progress bar.\tabularnewline
pageTitle & \texttt{"experiment"} & Page title.\tabularnewline
loadingFatalErrorMessage & \texttt{"..."} & The message shown when a
fatal error is encountered while loading chunks\tabularnewline
loadingNonfatalErrorMessage & \texttt{"..."} & The message shown when a
non-fatal error is encountered while loading chunks\tabularnewline
sendingResultsMessage & \texttt{"..."} & The message shown while results
are being sent to the server.\tabularnewline
completionMessage & \texttt{"..."} & The message shown when the results
are successfully sent to the server.\tabularnewline
completionErrorMessage & \texttt{"..."} & The message shown when there
is an error sending the results to the server.\tabularnewline
practiceItemTypes & \texttt{{[}{]}} & A list of types for practice
sentences. A sentence whose type is in this list will have a blue
``practice'' heading above it.\tabularnewline
practiceItemMessage & \texttt{"Practice"} & The message displayed for
practice items\tabularnewline
showOverview & false & See ``Overviews'' section\tabularnewline
centerItems & true & Whether or not items should be centered on the
page.\tabularnewline
equalGroupSizes & false & If true, groups in latin square designs are
required to contain equal numbers of items.\tabularnewline
manualSendResults & false & See ``Sending Results Early'' subsection of
``Shuffle Sequences'' section.\tabularnewline
\bottomrule
\end{longtable}

\subsection{\texorpdfstring{Configuring \texttt{js\_includes},
\texttt{data\_includes} and
\texttt{css\_includes}}{Configuring js\_includes, data\_includes and css\_includes}}\label{configuring-jsux5fincludes-dataux5fincludes-and-cssux5fincludes}

The directories \texttt{js\_includes}, \texttt{data\_includes} and
\texttt{css\_includes} directories contain JavaScript and CSS files that
are necessary for the running of an experiment. The most important of
these is the data file in \texttt{data\_includes} containing the list of
items for the experiment, but each controller also has its own file in
\texttt{js\_includes} (for example, the \texttt{DashedSentence}
controller lives in \texttt{dashed\_sentence.js}). Many controllers also
define some CSS classes in corresponding files in \texttt{css\_includes}
(e.g. \texttt{DashedSentence.css}). If you write your own controllers,
you need to put the JavaScript and CSS files in these directories.

When the webpage for an experiment is accessed, the server concatenates
all the files in \texttt{js\_includes} and serves them up as a single
file (ditto for \texttt{data\_includes} and \texttt{css\_includes}). You
can tell the server to ignore some of the files in \texttt{js\_includes}
and \texttt{css\_includes} by editing the variables
\texttt{JS\_INCLUDES\_LIST}, \texttt{DATA\_INCLUDES\_LIST} and
\texttt{CSS\_INCLUDES\_LIST} in \texttt{server\_conf.py}. (There is a
comment documenting how to do this.) You may wish to exclude JavaScript
and CSS files which are not used by your experiment in order to reduce
the size of the file that needs to be downloaded. Since the files in
both directories are named after the controllers with which they are
associated, it is easy to see which files are superfluous.

\textbf{Important:} the file \texttt{global\_main.css} is required by
all controllers.

\section{HTML Code}\label{html-code}

Ibex provides three methods of passing HTML code to the \texttt{Message}
and \texttt{Form} controllers\\(currently the only controllers which
have an \texttt{html} option).

\subsection{Method 1: HTML in a JavaScript
String}\label{method-1-html-in-a-javascript-string}

Pass a JavaScript string containing the HTML code.

\subsection{Method 2: HTML as a JavaScript data
structre}\label{method-2-html-as-a-javascript-data-structre}

Pass a JavaScript data structure representing the HTML Code. For
example, here is the representation of a \texttt{div} element containing
two paragraphs:

\begin{Shaded}
\begin{Highlighting}[]
    \NormalTok{[}\StringTok{"div"}\NormalTok{,}
        \NormalTok{[}\StringTok{"p"}\NormalTok{, }\StringTok{"This is the first paragraph."}\NormalTok{],}
        \NormalTok{[}\StringTok{"p"}\NormalTok{, }\StringTok{"This is the second paragraph."}\NormalTok{, }\StringTok{"Containing two text nodes."}\NormalTok{]}
    \NormalTok{]}
\end{Highlighting}
\end{Shaded}

Note that a space will automatically be inserted between the two text
nodes in the second paragraph. If you wanted to set the foreground color
of the \texttt{div} to red, you could use the following code:

\begin{Shaded}
\begin{Highlighting}[]
    \NormalTok{[[}\StringTok{"div"}\NormalTok{, \{}\DataTypeTok{style}\NormalTok{: }\StringTok{"color: red;"}\NormalTok{\}],}
        \NormalTok{[}\StringTok{"p"}\NormalTok{, }\StringTok{"This is the first paragraph."}\NormalTok{],}
        \NormalTok{[}\StringTok{"p"}\NormalTok{, }\StringTok{"This is the second paragraph."}\NormalTok{, }\StringTok{"Containing two text nodes."}\NormalTok{]}
    \NormalTok{]}
\end{Highlighting}
\end{Shaded}

If you want to set DOM properties directly, you can use a key beginning
with \href{mailto:"@"}{\nolinkurl{"@"}}:

\begin{Shaded}
\begin{Highlighting}[]
    \NormalTok{[[}\StringTok{"div"}\NormalTok{, \{}\StringTok{"@style.color"}\NormalTok{: }\StringTok{"red"}\NormalTok{\}],}
        \NormalTok{[}\StringTok{"p"}\NormalTok{, }\StringTok{"This is the first paragraph."}\NormalTok{],}
        \NormalTok{[}\StringTok{"p"}\NormalTok{, }\StringTok{"This is the second paragraph."}\NormalTok{, }\StringTok{"Containing two text nodes."}\NormalTok{]}
    \NormalTok{]}
\end{Highlighting}
\end{Shaded}

Elements (e.g. \texttt{\&ldquo;} -- a left double quote) are specified
as in the following example:

\begin{Shaded}
\begin{Highlighting}[]
    \NormalTok{[[}\StringTok{"div"}\NormalTok{, \{}\StringTok{"@style.color"}\NormalTok{: }\StringTok{"red"}\NormalTok{\}],}
        \NormalTok{[}\StringTok{"p"}\NormalTok{, }\StringTok{"This is the first paragraph."}\NormalTok{],}
        \NormalTok{[}\StringTok{"p"}\NormalTok{, }\StringTok{"This is the second paragraph."}\NormalTok{, }\StringTok{"Containing two text nodes."}\NormalTok{],}
        \NormalTok{[}\StringTok{"p"}\NormalTok{, }\StringTok{"Here's a paragraph where "}\NormalTok{, [}\StringTok{"&ldquo;"}\NormalTok{], }\StringTok{"this"}\NormalTok{, [}\StringTok{"&rdquo;"}\NormalTok{], }\StringTok{" is quoted."}\NormalTok{]}
    \NormalTok{]}
\end{Highlighting}
\end{Shaded}

Spaces are \emph{not} automatically inserted before and after elements.

\subsection{Method 3: HTML in a separate
file}\label{method-3-html-in-a-separate-file}

Place the HTML in a file in the \texttt{chunk\_includes} dir. Files in
this directory can be included as follows:

\begin{Shaded}
\begin{Highlighting}[]
    \NormalTok{\{ }\DataTypeTok{html}\NormalTok{: \{ }\DataTypeTok{include}\NormalTok{: }\StringTok{"file_name.html"} \NormalTok{\} \}}
\end{Highlighting}
\end{Shaded}

\section{Creating your own
controllers}\label{creating-your-own-controllers}

This is quite easy if you are familiar with JavaScript/DHTML and the
jQuery JavaScript library. (One of the changes in Ibex from webspr 0.2
is that controllers are now JQuery.ui widgets.) As an example, here's
the code for a simplified version of the \texttt{Message} controller:

\begin{Shaded}
\begin{Highlighting}[]
\FunctionTok{define_ibex_controller}\NormalTok{(\{}
\DataTypeTok{name}\NormalTok{: }\StringTok{"Message"}\NormalTok{,}

\DataTypeTok{jqueryWidget}\NormalTok{: \{}
    \DataTypeTok{_init}\NormalTok{: }\KeywordTok{function}\NormalTok{() \{}
        \CommentTok{// Boilerplate code that appears in all controllers.}
        \KeywordTok{this}\NormalTok{.}\FunctionTok{cssPrefix} \NormalTok{= }\KeywordTok{this}\NormalTok{.}\OtherTok{options}\NormalTok{.}\FunctionTok{_cssPrefix}\NormalTok{;}
        \KeywordTok{this}\NormalTok{.}\FunctionTok{utils} \NormalTok{= }\KeywordTok{this}\NormalTok{.}\OtherTok{options}\NormalTok{.}\FunctionTok{_utils}\NormalTok{;}
        \KeywordTok{this}\NormalTok{.}\FunctionTok{finishedCallback} \NormalTok{= }\KeywordTok{this}\NormalTok{.}\OtherTok{options}\NormalTok{.}\FunctionTok{_finishedCallback}\NormalTok{;}

        \KeywordTok{this}\NormalTok{.}\FunctionTok{html} \NormalTok{= }\KeywordTok{this}\NormalTok{.}\OtherTok{options}\NormalTok{.}\FunctionTok{html}\NormalTok{;}
        \KeywordTok{this}\NormalTok{.}\OtherTok{element}\NormalTok{.}\FunctionTok{addClass}\NormalTok{(}\KeywordTok{this}\NormalTok{.}\FunctionTok{cssPrefix} \NormalTok{+ }\StringTok{"message"}\NormalTok{);}
        \KeywordTok{this}\NormalTok{.}\OtherTok{element}\NormalTok{.}\FunctionTok{append}\NormalTok{(}\FunctionTok{htmlCodeToDOM}\NormalTok{(}\KeywordTok{this}\NormalTok{.}\FunctionTok{html}\NormalTok{));}

        \KeywordTok{this}\NormalTok{.}\FunctionTok{transfer} \NormalTok{= }\FunctionTok{dget}\NormalTok{(}\KeywordTok{this}\NormalTok{.}\FunctionTok{options}\NormalTok{, }\StringTok{"transfer"}\NormalTok{, }\StringTok{"keypress"}\NormalTok{);}
        \FunctionTok{assert}\NormalTok{(}\KeywordTok{this}\NormalTok{.}\FunctionTok{transfer} \NormalTok{== }\StringTok{"keypress"} \NormalTok{|| }\KeywordTok{typeof}\NormalTok{(}\KeywordTok{this}\NormalTok{.}\FunctionTok{transfer}\NormalTok{) == }\StringTok{"number"}\NormalTok{,}
               \StringTok{"Value of 'transfer' option of Message must either be the string 'keypress' or a number"}\NormalTok{);}

        \KeywordTok{if} \NormalTok{(}\KeywordTok{this}\NormalTok{.}\FunctionTok{transfer} \NormalTok{== }\StringTok{"keypress"}\NormalTok{) \{}
            \KeywordTok{var} \NormalTok{t = }\KeywordTok{this}\NormalTok{;}
            \CommentTok{// See below for info on 'safeBind'.}
            \KeywordTok{this}\NormalTok{.}\FunctionTok{safeBind}\NormalTok{(}\FunctionTok{$}\NormalTok{(document), }\StringTok{'keydown'}\NormalTok{, }\KeywordTok{function} \NormalTok{() \{}
                \OtherTok{t}\NormalTok{.}\FunctionTok{finishedCallback}\NormalTok{(}\KeywordTok{null}\NormalTok{);}
                \KeywordTok{return} \KeywordTok{true}\NormalTok{;}
            \NormalTok{\});}
        \NormalTok{\}}
        \KeywordTok{else} \NormalTok{\{}
            \CommentTok{// See below for info on 'this.utils'.}
            \KeywordTok{this}\NormalTok{.}\OtherTok{utils}\NormalTok{.}\FunctionTok{setTimeout}\NormalTok{(}\KeywordTok{this}\NormalTok{.}\FunctionTok{finishedCallback}\NormalTok{, }\KeywordTok{this}\NormalTok{.}\FunctionTok{transfer}\NormalTok{);}
        \NormalTok{\}}
    \NormalTok{\}}
\NormalTok{\},}

\DataTypeTok{properties}\NormalTok{: \{}
    \DataTypeTok{obligatory}\NormalTok{: [}\StringTok{"html"}\NormalTok{],}
    \DataTypeTok{countsForProgressBar}\NormalTok{: }\KeywordTok{false}\NormalTok{,}
    \DataTypeTok{htmlDescription}\NormalTok{: }\KeywordTok{function} \NormalTok{(opts) \{}
        \KeywordTok{var} \NormalTok{d = }\FunctionTok{htmlCodeToDOM}\NormalTok{(}\OtherTok{opts}\NormalTok{.}\FunctionTok{html}\NormalTok{);}
        \KeywordTok{return} \FunctionTok{truncateHTML}\NormalTok{(d, }\DecValTok{100}\NormalTok{);}
    \NormalTok{\}}
\NormalTok{\}}
\NormalTok{\});}
\end{Highlighting}
\end{Shaded}

The call to \texttt{ibex\_controller\_set\_properties} sets some
properties of the \texttt{Message} controller. The \texttt{obligatory}
option specifies those options which must obligatorily be given to the
controller. In this case, it is obligatory that the controller be given
an \texttt{"html"} option. The \texttt{obligatory} key must be present
in the object passed to \texttt{ibex\_controller\_set\_properties}. The
\texttt{countsForProgressBar} property is optional and is \texttt{true}
by default. It determines whether instances of the controller count
towards the size of the progress bar. (This option can be overriden on
an item-by-item basis by setting the \texttt{countsForProgressBar}
property of an item.) The \texttt{htmlDescription} function should
return either HTML or a DOM node which gives a brief summary of the
content of an instance of the controller for use in overviews.

The \texttt{options} dictionary for the widget has three special values
set:

\begin{itemize}
\itemsep1pt\parskip0pt\parsep0pt
\item
  A function (\texttt{\_finishedCallback}) which should be called with
  lines to be added to the results file when the controller is complete.
\item
  A \texttt{\_utils} object which contains some useful functions.
\item
  A \texttt{\_cssPrefix} string, which gives the controller its
  designated CSS prefix for all CSS class names/ids which it adds to DOM
  nodes (see section below, ``CSS mangling'').
\end{itemize}

In the case of \texttt{Message}, \texttt{\_finishedCallback} is called
with \texttt{null} as its argument\\because this controller does not add
any lines to the results file.\\In general, the format of a non-null
argument to \texttt{finishedCallback} is as follows:

\begin{Shaded}
\begin{Highlighting}[]
    \NormalTok{[}
        \CommentTok{// Line 1.}
        \NormalTok{[ [}\StringTok{"fieldname1"}\NormalTok{, value1], [}\StringTok{"fieldname2"}\NormalTok{, value2], [}\StringTok{"fieldname3"}\NormalTok{, value3], ... ],}
        \CommentTok{// Line 2.}
        \NormalTok{[ [}\StringTok{"fieldname2"}\NormalTok{, value2], [}\StringTok{"fieldname2"}\NormalTok{, value2], [}\StringTok{"fieldname3"}\NormalTok{, value3], ... ],}
        \NormalTok{...}
    \NormalTok{]}
\end{Highlighting}
\end{Shaded}

As can be seen in the code for \texttt{Message}, the \texttt{utils}
object provides a \texttt{setTimeout} method similar to the builtin
\texttt{setTimeout} function of JavaScript. Any timeouts set using this
method are automatically cleared when the controller is complete.
Similarly, a \texttt{safeBind} method has been added to jQuery, which
ensures that event handlers are automatically unregistered once a
controller instance is finished.

\subsection{CSS mangling}\label{css-mangling}

Every class/id name in a CSS file in the \texttt{css\_includes}
directory is prepended with the string `FILENAME-'. The CSS for each
controller should live in \texttt{CONTROLLER\_NAME.css}. This ensures
that there are no namespace clashes between controllers. Note that if
you wish to prevent automatic CSS mangling (e.g.~because of a bug in the
mangling code that you need to work around), you may prefix your CSS
file name with \texttt{"global\_"}. This causes the file to be served
as-is.

\section{Overviews}\label{overviews}

Sometimes it's useful to get an overview of the sequence of items in an
experiment without actually running through each item. There are two
ways of getting webspr to display an overview of this sort. The first is
to add the statement \texttt{var\ showOverview\ =\ true;} to your data
file. The second is to go to the page \texttt{overview.html} instead of
\texttt{experiment.html}.

\section{Cross-browser compatibility}\label{cross-browser-compatibility}

I have tested compatibility with the following browsers:

\begin{itemize}
\itemsep1pt\parskip0pt\parsep0pt
\item
  Internet Explorer 6 and 7. (Note that support for IE 5 has been
  dropped as of version 0.3 of this software, since the jQuery library
  doesn't support this browser.)
\item
  Firefox 1, 1.5, 2, and 3.
\item
  Safari 3.
\item
  Opera 9.
\item
  Google Chrome.
\end{itemize}

Known cross-browser issues:

\begin{itemize}
\item
  Overviews (see previous section) currently do not work on all versions
  of Internet Explorer.
\item
  Pressing `6' and `7' in Opera affects text size; this interacts badly
  with acceptability judgments on scales including 6 and 7. I have not
  yet found a way of preventing Opera from interpreting these
  keypresses.
\item
  The stand-alone server serves up the JavaScript and CSS include files
  with a \texttt{Pragma:\ nocache} so that any changes you make will be
  immediately reflected if you refresh \texttt{experiment.html}.
  However, some versions of Internet Explorer ignore this pragma (I
  think this is a bug in IE, but not 100\% sure yet), so you will need
  to delete your temporary internet files and then refresh. This is not
  such a big issue for live experiments, but it makes developing and
  testing experiments using IE a big PITA. The obvious workaround is
  simply not to use IE for these purposes.
\end{itemize}

\section{Terminological
clarifications}\label{terminological-clarifications}

Unfortunately, the terms I've used relating to the design of experiments
(latin squares, etc.) are confusing as they use some non-standard terms,
and make non-standard use of some standard terms. Specifically:

\begin{itemize}
\item
  The term `item number' is used to describe the number assigned to a
  controller based on its position in the list of controllers in the
  data file. Of course, controllers are not normally in one-to-one
  correspondence with `items' in the usual sense of the term.
\item
  The term `group' is used to refer to what are usually called items
  (i.e.~sets of conditions).
\end{itemize}

\section{Known problems and issues}\label{known-problems-and-issues}

The following are some subtle problems which can often arise when
writing a\\data file:

\begin{itemize}
\item
  A missing comma in the list of sentences in data.js can cause highly
  obscure JavaScript errors with no obvious relation to their source.
\item
  Some browsers accept trailing commas in JavaScript array literals
  (i.e.~they accept {[}1,2,3,4,{]} as a fine array); others do not. If
  your browser of choice accepts array literals of this form, be sure to
  check that your data.js has no trailing commas so that there will be
  no browser incompatibilities. It is quite easy to introduce trailing
  commas by accident if you comment out some of the items in the
  \texttt{items} array.
\end{itemize}

Most browsers allow strings to be indexed using square brackets (i.e.
\texttt{"foo"``{[}1{]}} \texttt{==\ "f"}). Internet Explorer, however,
requires the use of the string's \texttt{charAt} method.

For debugging, I recommend using Firefox's JavaScript console. Most
syntax errors in a data file will result in an alert popping up with a
warning that the \texttt{items} array has not been defined. You can
usually get a much more informative error message by looking at the
JavaScript console.

\section{Changes}\label{changes}

Changes between 0.3.8 (current version and 0.3.7):

\begin{itemize}
\item
  Fix bug with VBox controller which caused transition to next item to
  fail sometimes.
\item
  Fix bug in \texttt{Form} controller which prevented the
  \texttt{continueOnReturn} option

  \begin{quote}
  \begin{quote}
  from taking effect (thanks to Laurel Brehm).
  \end{quote}
  \end{quote}
\item
  Some preliminary support for playing sounds (not yet documented).
\end{itemize}

Changes between 0.3.7 (current version) and 0.3.6:

\begin{itemize}
\itemsep1pt\parskip0pt\parsep0pt
\item
  Fix bug which prevents \texttt{countsForProgressBar} option taking
  effect.
\end{itemize}

Changes between 0.3.6 and 0.3.5:

\begin{itemize}
\itemsep1pt\parskip0pt\parsep0pt
\item
  Improve loading of files in chunk\_includes. Chunks are now preloaded
  so that experiments will not be disrupted.
\end{itemize}

Changes between 0.3.5 and 0.3.4:

\begin{itemize}
\itemsep1pt\parskip0pt\parsep0pt
\item
  Fix a bug in the recording of answers for certain options of the
  \texttt{Question} controller (thanks to Michael Yoshitaka Erlewine).
\item
  Fix a bug in VBox which caused errors to appear on the console in some
  cases (thanks to Alexandre Cremers).
\end{itemize}

Changes between 0.3.4 (current version) and 0.3.3:

\begin{itemize}
\itemsep1pt\parskip0pt\parsep0pt
\item
  Fix a bug which caused commas to be unescaped in `Form' controller
  results.
\item
  Fix a minor bug in server.py (most likely did not have any
  user-visible effects).
\end{itemize}

Changes between 0.3.3 and 0.3.2:

\begin{itemize}
\itemsep1pt\parskip0pt\parsep0pt
\item
  Improve recording of HTML sentences in results file with
  \texttt{AcceptabilityJudgment} and \texttt{FlashSentence}.
\item
  Fix a venerable bug in server.py which caused files to be served with
  a leading blank line with the toy server.
\item
  Remove `md5' option for `sentenceDescType'.
\end{itemize}

Changes between 0.3.2 and 0.3.1:

\begin{itemize}
\itemsep1pt\parskip0pt\parsep0pt
\item
  Corrected error in docs (``requiresConsent'' to ``consentRequired'').
\item
  Made it possible to use HTML with \texttt{FlashSentence} controller
  (and hence also with \texttt{AcceptabilityJudgment}). Currently
  undocumented; see Google group discussion.
\end{itemize}

Changes between 0.3.1 and 0.3.0:

\begin{itemize}
\itemsep1pt\parskip0pt\parsep0pt
\item
  Fixed a regression introduced in 0.3.0 (trailing comma stopped
  experiments working in IE).
\end{itemize}

Changes between 0.3.0 and 0.3-beta19:

\begin{itemize}
\itemsep1pt\parskip0pt\parsep0pt
\item
  Fixed a bug involving timeouts with the \texttt{AcceptabilityJudgment}
  controller.
\end{itemize}

Changes between 0.3-beta19 and 0.3-beta18:

\begin{itemize}
\itemsep1pt\parskip0pt\parsep0pt
\item
  Added special \texttt{\_\_SetCounter\_\_} controller (see comment in
  \texttt{example\_data.js}).
\end{itemize}

Changes between 0.3-beta18 and 0.3-beta17:

\begin{itemize}
\itemsep1pt\parskip0pt\parsep0pt
\item
  Add option to save reaction times to \texttt{Form} controller.
\end{itemize}

Changes between 0.3-beta17 and 0.3-beta16:

\begin{itemize}
\itemsep1pt\parskip0pt\parsep0pt
\item
  Fix a bug which caused timings for question answers to be recorded
  incorrectly in some instances.
\item
  Add the \texttt{presentHorizontally} option to the \texttt{Question}
  controller.
\item
  Add the option to assign keys to answers based on the (random) order
  in which they are presented.
\end{itemize}

Changes between 0.3-beta16 and 0.3-beta15:

\begin{itemize}
\itemsep1pt\parskip0pt\parsep0pt
\item
  Fix bug which stopped ibex working on Internet Explorer.
\end{itemize}

Changes between 0.3-beta15 and 0.3-beta14:

\begin{itemize}
\itemsep1pt\parskip0pt\parsep0pt
\item
  Added the \texttt{DashedAcceptabilityJudgment} controller for easy
  speeded acceptability judgments.
\item
  Cleaned up the way controllers are defined (not a user-visible
  change).
\item
  Fixed a bug in the file-locking code in \texttt{server.py}.
\item
  Fixed a bug with result concatenation in the \texttt{VBox} controller
  which surface on Chrome/Safari.
\item
  Make it possible to create \texttt{Form} and \texttt{Message}
  controllers which can't be completed (useful in VBoxes).
\end{itemize}

Changes between 0.3-beta14 and 0.3-beta13:

\begin{itemize}
\itemsep1pt\parskip0pt\parsep0pt
\item
  It is now possible to set the \texttt{countsForProgressBar} option at
  the level of individual items.
\item
  Add `continueOnReturn' option to the \texttt{Form} controller.
\item
  Add the option to display all words at the center of the screen with
  the \texttt{DashedSentence} controller.
\item
  Automatic generation of \texttt{experiment.html} and
  \texttt{overview.html} can now deal with installations where
  \texttt{sever.py} has been renamed.
\item
  Upgrade to jquery 1.5.1 and jquery-ui 1.8.10
\item
  Minor bug fixes.
\end{itemize}

Changes between 0.3-beta13 and 0.3-beta12:

\begin{itemize}
\itemsep1pt\parskip0pt\parsep0pt
\item
  Ibex no longer depends on cookies to set the value of the latin square
  counter.
\item
  Trailing comma removed in example\_data.js.
\item
  \texttt{DashedSentence} controller now uses spans instead of divs for
  the individual words. This should make things work more smoothly with
  right-to-left languages (though I haven't tested this yet).
\item
  Highlighting text no longer shows hidden words when using
  \texttt{DashedSentence}.
\item
  Fixed regression introduced in 0.3 where line breaks in SPR sentences
  did not get recorded in the results.
\end{itemize}

Changes between 0.3-beta12 and 0.3-beta11:

\begin{itemize}
\itemsep1pt\parskip0pt\parsep0pt
\item
  Fix a minor bug which caused `DYNAMIC' to be used instead of `NULL' in
  some cases in the results file.
\item
  Fixed a bug which occasionally led to inconsistent values for the
  unique user ID in the results file (second column in each line).
\end{itemize}

Changes between 0.3-beta11 and 0.3-beta10:

\begin{itemize}
\itemsep1pt\parskip0pt\parsep0pt
\item
  Added the \texttt{modifyRunningOrder} config option.
\item
  \texttt{example\_data.js} has been slimmed down and reformatted.
\end{itemize}

Changes between 0.3-beta10 and 0.3-beta9:

\begin{itemize}
\itemsep1pt\parskip0pt\parsep0pt
\item
  Time is now recorded for the last word in a sentence in self-paced
  reading experiments.
\end{itemize}

Changes between 0.3-beta9 and 0.3-beta8:

\begin{itemize}
\itemsep1pt\parskip0pt\parsep0pt
\item
  It is now possible to control the point in the experiment at which
  results are sent to the server using the special
  \texttt{\_\_SendResults\_\_} controller.
\item
  Upgrade from jQuery 1.4.0 to 1.4.2.
\item
  Fix bug which prevented ``keypress'' transfer working with the
  \texttt{Separator} controller.
\item
  Fix bug which prevented the \texttt{DashedSentence} controller from
  being able to display text one ``chunk'' at a time.
\item
  Minor fixes.
\end{itemize}

Changes between 0.3-beta8 and 0.3-beta7:

\begin{itemize}
\itemsep1pt\parskip0pt\parsep0pt
\item
  Fix bug which caused self-paced reading experiments to display
  incorrectly in IE 8.
\item
  Fix bug which caused question answers to appear without numbers in IE.
\end{itemize}

Changes between 0.3-beta7 and 0.3-beta6:

\begin{itemize}
\itemsep1pt\parskip0pt\parsep0pt
\item
  Fix bug which caused questions not to be recorded in results file.
\item
  Checkboxes can be specified as obligatory in the \texttt{Form}
  controller.
\item
  Minor changes to example data file.
\end{itemize}

Changes between 0.3-beta6 and 0.3-beta5:

\begin{itemize}
\itemsep1pt\parskip0pt\parsep0pt
\item
  Fix bug in VBox.js which caused some things to appear off center.
  (This bug only affected the presentation of items, not the results.)
\end{itemize}

Changes between 0.3-beta5 and 0.3-beta4:

\begin{itemize}
\item
  Improvements to the recording of times in self-paced reading
  experiments. Times should now be recorded a little more accurately.
\item
  DashedSentence controller works on the iPhone (though it's not
  practical for actually doing experiments).
\item
  Option to minify JavaScript code in \texttt{data\_includes} dir
  (turned off by default).
\item
  In the results file, the MD5 hash of the subject's IP address has been
  replaced with a hash based on the IP address together with a variety
  of other identifying information (e.g.~user agent string, subject's
  screen size). This both provides a better unique identifier, and is
  preferable from a privacy point of view. (It isn't difficult to
  reverse an IP address hash, but it would be pretty difficult to
  reverse the new hashes.)
\end{itemize}

Changes between 0.3-beta4 and 0.3-beta3:

\begin{itemize}
\itemsep1pt\parskip0pt\parsep0pt
\item
  Fix a bug in the \texttt{Form} controller that caused values for radio
  buttons to be incorrectly recorded in the results file.
\end{itemize}

Changes between 0.3-beta3 and 0.3-beta2:

\begin{itemize}
\itemsep1pt\parskip0pt\parsep0pt
\item
  \texttt{Form} controller added for collecting data from participants.
\item
  New method of passing HTML to the \texttt{Message} and \texttt{Form}
  controllers (HTML files can be included from \texttt{chunk\_includes}
  dir).
\end{itemize}

Changes between 0.3-beta2 and 0.3-beta1:

\begin{itemize}
\itemsep1pt\parskip0pt\parsep0pt
\item
  Bug fixes.
\end{itemize}

Changes between 0.3-beta1 and 0.2.7:

\begin{itemize}
\itemsep1pt\parskip0pt\parsep0pt
\item
  Name changed to ``Ibex'' (Internet Based EXperiments).
\item
  Removed dependence on \texttt{paste} module for stand-alone server.
  The server now requires only modules that are bundled with Python
  \textgreater{}= 2.3.
\item
  JavaScript code now uses jQuery 1.4. Controllers are jQuery.ui
  widgets.
\item
  New ways of configuring the server. The server can now issue an HTTP
  GET request to get its config variables. Config variables can be
  inlined in server.py as well as being given in the separate
  \texttt{server\_conf.py} file. These features are used for the Ibex
  Farm (\url{http://spellout.net/ibexfarm}).
\item
  CSS files in css\_includes now modified automatically by the server to
  ensure that there are no CSS namespace clashes between different
  widgets.
\item
  Server now caches css and js includes (i.e.~the results of requests
  like server.py?includes=js).
\item
  Support for IE 5.5 dropped (jQuery does not support IE 5.5).
\item
  Some additional options for the \texttt{DashedSentence} controller.
\end{itemize}

\end{document}
